
\chapter{Einleitung}

\section{Problemstellung}
Haben Sie sich schon einmal gefragt, wie man ein Computerspiel von Grund auf entwickelt?
\\ In dieser Diplomarbeit erforschen wir den Prozess der Spieleentwicklung mit Hilfe der Unity-Engine und betrachten detailliert den Entwicklungsprozess eines Spiels. Von der Idee bis zur Fertigstellung des Spiels teilen wir unsere eigenen Erfahrungen und Herausforderungen während der Entwicklung. Dadurch erhalten Sie einen Einblick in die verschiedenen Aspekte und Entscheidungen, die bei der Erschaffung eines Computerspiels berücksichtigt werden müssen.

\section{Wirtschaftliche Begründung für die Entwicklung eines Spiels}
Der Gaming-Markt ist ein Milliardengeschäft. Allein in der D-A-CH-Region wird im Jahr 2023 ein Umsatz von rund 11,10 Milliarden Euro erwartet. Die Gaming-Branche bietet also ein enormes wirtschaftliches Potenzial.

\subsection{Die zunehmende Digitalisierung des Marktes}
Die Digitalisierung hat den Vertrieb von Spielen revolutioniert. Der Verkauf physischer Kopien geht zurück, während der digitale Vertrieb immer wichtiger wird. Dies ermöglicht eine breitere Verbreitung der Spiele und eine schnellere Verbindung mit den Spielern in der D-A-CH-Region.

\subsection{Die hohe Nutzerzahl als Umsatztreiber}
Prognosen zufolge werden im Jahr 2027 rund 42,77 Millionen Nutzer im Markt Videospiele in der D-A-CH-Region aktiv sein. Diese große Nutzerbasis bietet ein enormes Potenzial für Umsatzgenerierung. In-Game-Käufe, Abonnementmodelle und Werbung sind zusätzliche Einnahmequellen, von denen eine solche Nutzerbasis profitieren kann.

\subsection{Technologische Fortschritte}
Dank fortschreitender Technologie sind immer anspruchsvollere Spielerlebnisse möglich. Technologien wie Virtual Reality (VR) und Augmented Reality (AR) eröffnen völlig neue Möglichkeiten für die Spieleentwicklung und schaffen einzigartige Spielerlebnisse, die die Nutzer in der D-A-CH-Region begeistern.

\subsection{Chancen in einem wachsenden Markt}
Die Entwicklung eines eigenen Spiels bietet nicht nur hohe Umsatzpotenziale, sondern auch die Chance, Teil einer dynamischen und innovativen Industrie zu sein. Die Gaming-Branche bietet Raum für Kreativität und ermöglicht es Ihnen, Ihre eigene Vision und Ideen in ein erfolgreiches und unterhaltsames Produkt umzusetzen.

\pagebreak

\section{Vergleich zwischen Gaming und Video-Streaming-Services in der D-A-CH-Region}

\subsection{Umsatz}

Der Gaming-Markt generiert in Bezug auf Umsatz größere Zahlen als der Markt für Video-Streaming-Services. Im Jahr 2023 wird der Umsatz im Markt Video-Streaming-Services (SVoD) voraussichtlich rund 4,28 Milliarden Euro betragen, während der Umsatz im Markt Videospiele etwa 11,10 Milliarden Euro erreichen wird.

\subsection{Umsatzwachstum}

Das jährliche Umsatzwachstum im Markt Video-Streaming-Services (SVoD) wird für das Jahr 2027 auf 9,65 prognostiziert, während das jährliche Umsatzwachstum im Markt Videospiele bei 7,42 liegt. Beide Märkte verzeichnen solides Wachstum, wobei der Video-Streaming-Services-Markt ein etwas höheres Wachstum aufweist.

\subsection{Nutzerzahl}

Die Anzahl der Nutzer spielt eine Rolle bei der Betrachtung der beiden Märkte. Prognosen zufolge werden im Jahr 2027 etwa 49,94 Millionen Nutzer Video-Streaming-Services (SVoD) nutzen, während im Markt Videospiele 42,77 Millionen Nutzer erwartet werden. Der Video-Streaming-Services-Markt hat eine größere Nutzerbasis, aber der Unterschied ist nicht signifikant.

\subsection{Durchschnittlicher Erlös pro Nutzer (ARPU)}

Der durchschnittliche Erlös pro Nutzer (ARPU) liegt im Jahr 2023 bei rund 105,70 Euro im Markt Video-Streaming-Services (SVoD) und bei 281,60 Euro im Markt Videospiele. Im Gaming-Markt können also pro Nutzer höhere Umsätze erzielt werden.

\section{Attraktive Chancen für die Entwicklung eines eigenen Spiels}

Trotz des Wachstums und Potenzials beider Märkte bietet die Gaming-Branche aufgrund ihrer höheren Umsätze und der Möglichkeit, ein breites Publikum anzusprechen, attraktive Chancen für die Entwicklung eines eigenen Spiels in der D-A-CH-Region.