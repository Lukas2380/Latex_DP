
\chapter{Introduction}

\section{Problem Setting}
Haben Sie sich jemals gefragt, wie man von Null auf ein Computerspiel baut? Diese Frage haben wir uns im dritten und vierten Semester immer wieder gestellt. In dieser Diplomarbeit werden wir den Prozess hinter der Entwicklung eines Spiels mithilfe der Unity-Engine erforschen und darstellen. Von der Idee bis zur Fertigstellung des Spiels werden wir unsere Erfahrungen und auch unsere Fehltritte während der Entwicklung genauer erklären und bildlich schildern.

\section{Wirtschaftliche Begründung für die Entwicklung eines Spiels}
Der Gaming-Markt boomt und generiert weltweit Milliarden Euro an Umsatz. Im Jahr 2023 wird der Umsatz im Markt Videospiele voraussichtlich etwa 375,50 Mrd. Euro betragen (Statista). Diese beeindruckenden Zahlen verdeutlichen das immense wirtschaftliche Potenzial der Gaming-Branche.

\subsection{Die zunehmende Digitalisierung des Marktes}
Die Digitalisierung hat den Vertrieb von Spielen revolutioniert. Der Verkauf physischer Kopien geht zurück, während der digitale Vertrieb stark an Bedeutung gewinnt. Dies ermöglicht eine breitere Verbreitung der Spiele und eine schnellere Verbindung mit den Spielern weltweit.

\subsection{Die hohe Nutzerzahl als Umsatztreiber}
Im Jahr 2027 werden laut Prognose etwa 3,10 Mrd. Nutzer im Markt Videospiele aktiv sein (Statista). Diese große Nutzerbasis bietet ein enormes Potenzial für Umsatzgenerierung. In-Game-Käufe, Abonnementmodelle und Werbung sind zusätzliche Einnahmequellen, die von einer solchen Nutzerbasis profitieren können.

\subsection{Technologieentwicklung für immersive Spielerlebnisse}
Die stetige technologische Entwicklung ermöglicht immer anspruchsvollere und immersive Spielerlebnisse. Technologien wie Virtual Reality (VR) und Augmented Reality (AR) eröffnen völlig neue Möglichkeiten für die Spieleentwicklung und schaffen einzigartige Spielerlebnisse, die die Nutzer begeistern.

\subsection{Chancen in einem florierenden Markt}
Die wirtschaftliche Begründung für die Entwicklung eines Spiels liegt nicht nur in den beeindruckenden Umsatzzahlen, sondern auch in der Chance, Teil einer dynamischen und innovativen Industrie zu sein. Die Gaming-Branche bietet Raum für Kreativität und ermöglicht es Ihnen, Ihre eigene Vision und Ideen in ein erfolgreiches und unterhaltsames Produkt umzusetzen.

\section{Erforschung des Entwicklungsprozesses}
In dieser Diplomarbeit werden wir den Entwicklungsprozess eines Spiels mithilfe der Unity-Engine detailliert betrachten. Von der Idee bis zur Fertigstellung des Spiels werden wir unsere eigenen Erfahrungen und Herausforderungen während der Entwicklung teilen. Dies bietet einen Einblick in die vielfältigen Aspekte und Entscheidungen, die bei der Erschaffung eines Computerspiels berücksichtigt werden müssen.

Durch die Kombination von Kreativität, technischem Know-how und einer fundierten wirtschaftlichen Begründung bietet die Entwicklung eines Computerspiels eine spannende Möglichkeit, Teil einer aufstrebenden Branche zu sein und ein Produkt zu schaffen, das Millionen von Spielern auf der ganzen Welt begeistert.

\section{Problem Statement}

\section{Inventions OR Solutions OR Contributions}

\section{Structure of this Thesis}