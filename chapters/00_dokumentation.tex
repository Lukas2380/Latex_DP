\chapter*{Dokumentation}
\addcontentsline{toc}{chapter}{Dokumentation}

\renewcommand{\arraystretch}{2} % Anpassen der Zeilenhöhe

\begin{tabular}{|m{0.3\textwidth}|m{0.7\textwidth}|}
\hline
Namen der Verfasser/innen & Lukas Schachinger, Martin Usta \\
\hline
Jahrgang & 5AAIFT / 6AAIFT \\
\hline
Thema der Diplomarbeit & Das Thema der Diplomarbeit ist Unity Game Design und Development. Es wird mithilfe der Dokumentation und eines spielbaren Prototyps gezeigt, wie man ein Spiel entwickelt. 
In der Dokumentation werden die Vorgehensweisen und die Programme, welche verwendetet werden beschrieben. \\
\hline
Kooperationspartner & / \\
\hline
\end{tabular}

\vspace{10pt}

\noindent
\begin{tabular}{|m{0.3\textwidth}|m{0.7\textwidth}|}
\hline
Aufgabenstellung & Teilaufgabe von Lukas Schachinger: \newline \newline Programmierung der physikalischen Grundfunktionen und Implementierung und Design einer grafischen Benutzeroberfläche \newline \newline Teilaufgabe von Martin Usta: \newline \newline Aufbau der 3D Spielewelt und Design der verschiedenen Level. Programmierung und Design der aktiven Spielfiguren. Implementierung der Musik und Soundeffekte der Spielumgebung.\\
\hline
\end{tabular}

\pagebreak

\noindent
\begin{tabular}{|m{0.3\textwidth}|m{0.7\textwidth}|}
\hline
Realisierung & Die Spielobjekte werden mithilfe von Blender designet und modelliert. \newline \newline Der Prototyp wird in Unity realisiert. Die fertigen Assets werden in Unity integriert und dann in das Spiel eingebaut. \newline \newline In C\# werden die Skripte geschrieben und dann an die Spielobjekte gebunden. \\
\hline
\end{tabular}

\vspace{10pt}

\noindent
\begin{tabular}{|m{0.3\textwidth}|m{0.7\textwidth}|}
\hline
Ergebnisse & Das Ergebnis wird ein spielbarer Prototyp sowie eine umfassende Dokumentation zum Thema Spieleentwicklung. Es soll mithilfe der Dokumentation erkennbar sein wie der Prototyp geplant und entwickelt wurde.  \\
\hline
\end{tabular}

\pagebreak

\noindent
\begin{tabular}{|m{0.3\textwidth}|m{0.7\textwidth}|}
\hline
Typische Grafik, Foto etc. (mit Erläuterung) & \\
\hline
\end{tabular}

\vspace{10pt}

\noindent
\begin{tabular}{|m{0.3\textwidth}|m{0.7\textwidth}|}
\hline
Teilnahme an Wettbewerben, Auszeichnungen & Keine \\
\hline
\end{tabular}

\vspace{10pt}

\noindent
\begin{tabular}{|m{0.3\textwidth}|m{0.7\textwidth}|}
\hline
Möglichkeiten der Einsichtnahme in die Arbeit & Im Archiv der Abteilung Elektronik und Technische Informatik der HTL Mödling \\
\hline
\end{tabular}

\vspace{10pt}

\noindent
\begin{tabular}{|m{0.325\textwidth}|m{0.325\textwidth}|m{0.325\textwidth}|}
\hline
Approbation (Datum / Unterschrift) & {\tiny Prüfer/Prüferin} \newline \newline & {\tiny Direktor/Direktorin} \newline {\tiny Abteilungsvorstand/Abteilungsvorständin} \newline \newline \\
\hline
\end{tabular}

\pagebreak

%here english ----------------------------------------------------------------

\noindent
\begin{tabular}{|m{0.3\textwidth}|m{0.7\textwidth}|}
\hline
Authors & Lukas Schachinger, Martin Usta \\
\hline
Academic Year & 5AAIFT / 6AAIFT \\
\hline
Topic & The topic of the diploma thesis is Unity Game Design and Development. It demonstrates the process of developing a game through documentation and a playable prototype. The documentation describes the methodologies and programs used. \\
\hline
Collaboration Partners & / \\
\hline
\end{tabular}

\vspace{10pt}

\noindent
\begin{tabular}{|m{0.3\textwidth}|m{0.7\textwidth}|}
\hline
Assignment & Subtask by Lukas Schachinger: \newline \newline Programming of the physics functions and implementation and design of a graphical user interface \newline \newline Subtask by Martin Usta: \newline \newline Construction of the 3D game world and design of various levels. Programming and design of active game characters. Implementation of music and sound effects in the game environment.\\
\hline
\end{tabular}

\pagebreak

\noindent
\begin{tabular}{|m{0.3\textwidth}|m{0.7\textwidth}|}
\hline
Implementation & The game objects are designed and modeled using Blender. \newline \newline The prototype is implemented in Unity. The finished assets are integrated into Unity and then incorporated into the game. \newline \newline The scripts are written in C\# and then bound to the game objects. \\
\hline
\end{tabular}

\vspace{10pt}

\noindent
\begin{tabular}{|m{0.3\textwidth}|m{0.7\textwidth}|}
\hline
Results & The result is a playable prototype and a detailed documentation on game development. The documentation should demonstrate how the prototype was planned and developed.  \\
\hline
\end{tabular}

\pagebreak

\noindent
\begin{tabular}{|m{0.3\textwidth}|m{0.7\textwidth}|}
\hline
Typical Graphics, Photos, etc. (with explanation) & \\
\hline
\end{tabular}

\vspace{10pt}

\noindent
\begin{tabular}{|m{0.3\textwidth}|m{0.7\textwidth}|}
\hline
Participation in Competitions, Awards & None \\
\hline
\end{tabular}

\vspace{10pt}

\noindent
\begin{tabular}{|m{0.3\textwidth}|m{0.7\textwidth}|}
\hline
Accessibility of \newline Diploma Thesis & Stored in the archive of the secondary technical college of Moedling, department of electronics and computer engineering \\
\hline
\end{tabular}

\vspace{10pt}

\noindent
\begin{tabular}{|m{0.325\textwidth}|m{0.325\textwidth}|m{0.325\textwidth}|}
\hline
Approval (Date / Signature) & {\tiny Examiner \newline \newline} & {\tiny Head of College / Department \newline \newline} \\
\hline
\end{tabular}