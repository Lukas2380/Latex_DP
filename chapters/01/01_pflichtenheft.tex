\pagebreak
\setauthorname{Lukas Schachinger \& Martin Usta}
\chapter{Pflichtenheft}
\section{Zielbestimmung}

Es soll eine spielbare Demo erstellt werden. Dabei wird dem Spieler ein Charakter zur Verfügung gestellt, mit dem er sich in einer Spielwelt bewegen kann. Diese Spielwelt besteht aus verschiedenen selbst entworfenen Leveln. Es wird eine Benutzeroberfläche entwickelt, die leicht verständliche Spielinformationen präsentiert.

Zusätzlich wird eine Dokumentation erstellt, in der beschrieben wird, wie der Prototyp entwickelt wurde und welche Programme dabei verwendet wurden.

\subsection{Musskriterien}
Die Musskriterien beschreiben die Punkte, welche definitiv ein teil des Spieles und der Dokumentation sein sollen. Die Musikprogramme wurden zwar außerhalb der Dokumentation evaluiert, aber entzifferten sich als weniger wichtig als die verbesserte künstliche Intelligenz von den Wunschkriterien.

\begin{itemize}
  \item Spiele-Demo
  \begin{itemize}
    \item Spieler-Charakter
    \item Spielewelt
    \item Level-Design
  \end{itemize}
  \item Evaluierung von Programmen zur Spieleentwicklung
  \begin{itemize}
    \item Design: Blender, Material Design (selbstgestaltet/Download aus dem Internet)
    \item Game Engine: Unity
    \item Musik: (Musik-Software)
  \end{itemize}
\end{itemize}

\pagebreak

\subsection{Wunschkriterien}
Die Wunschkriterien beschreiben Punkte, welche dem Prototypen hinzugefügt werden könnten aber nicht erforderlich sind. Ein Multiplayer wurde nicht eingebaut, aber die verbesserte künstliche Intelligenz schien eine sinnfolle addition zu dem Prototypen. 

\begin{itemize}
  \item Multiplayer
  \item Verbesserte künstliche Intelligenz (AI)
\end{itemize}

\subsection{Abgrenzungskriterien}
Die folgenden Punkte gelten als Abgrenzungskriterien. Das bedeutet, dass diese Punkte nicht teil der Arbeit sind und diese daher nicht behandelt oder gemacht werden.

\begin{itemize}
  \item Kein vollständiges Spiel
  \item Kein VR-Spiel
\end{itemize}

\section{Projektumfeldanalyse}
Diese Arbeit baut nicht auf bestehenden Projekten auf.

\chapter{Projektplan}
\section{Gesamtprojektplan}

\newcolumntype{C}[1]{>{\centering\arraybackslash}p{#1}}

\noindent
\begin{tabular}{|C{3cm}|C{3cm}|C{3cm}|C{3cm}|C{3cm}|}
\hline
\multicolumn{5}{|c|}{\cellcolor{green!20} Projekt} \\
\hline
\cellcolor{green!20}Analyse des Umfelds/Projektplanung & & & & \\
\hline
& \cellcolor{green!20} Erstellung des Grundgerüsts für den Prototyp & & & \\
\hline
& & \cellcolor{green!20} Fertigstellung der UI & & \\
\hline
& & & \cellcolor{green!20} Fertigstellung des Prototyps & \\
\hline
& & & & \cellcolor{green!20} Diplomarbeit Fertigstellen \\
\hline
\end{tabular}


\vspace{10pt}

\noindent
\begin{tabular}{|C{2.43cm}|C{2.43cm}|C{2.43cm}|C{2.43cm}|C{2.43cm}|C{2.43cm}|C{2.43cm}|}
    Projektstart & Meilenstein 1 Ende & Meilenstein 2 Ende & Meilenstein 3 Ende & Meilenstein 4 Ende & Meilenstein 5 Ende \\
    01.09.2022 & 30.11.2022 & 18.12.2022 & 28.02.2023 & 31.05.2023 & 01.09.2023 \\
\end{tabular}


\pagebreak

\section{Meilensteine}
\subsection{Analyse des Umfelds / Projektplanung}
Zeitraum: 01.09.2022 – 30.11.2022\\\\
Tätigkeiten:
\begin{itemize}
    \item Recherche der Programme
    \item Recherche der Spiel-Engines
    \item Umfeldanalyse Unity gegen Unreal Engine
    \item Umfeldanalyse der unterschiedlichen Modellierprogramme
    \item Kategorisierung der Projektschwerpunkte
    \item Zeiteinteilung der Projektschwerpunkte
    \item Einrichtung des Projektmanagementtools 
\end{itemize}

\subsection{Erstellung des Grundgerüsts für den Prototyps}
Zeitraum: 30.11.2022 – 18.12.2022\\\\
Tätigkeiten:
\begin{itemize}
    \item Erstellung der Low-Poly Assets
    \item Design und Aufbau des ersten Levels 
    \item Implementierung der Charaktersteuerung
    \item Implementierung der beweglichen Plattformen
    \item Design und Integration der ersten Spielfiguren
\end{itemize}
\pagebreak

\subsection{Fertigstellung des UI}
Zeitraum 18.12.2022 – 28.02.2023\\\\
Tätigkeiten:
\begin{itemize}
    \item Planung des UI für eine optimale User Experience
    \item Aufbau und Erstellung einer Menüführung
    \item Erstellung einer Statistikoberfläche
\end{itemize}

\subsection{Fertigstellen des Prototyps}
Zeitraum 28.02.2023 – 31.05.2023\\\\
Tätigkeiten:
\begin{itemize}
    \item Bugs fixen
    \item Implementierung der High-Poly Assets
    \item Fertigstellung der Designaufgaben
    \item Fertigstellung der Programmieraufgaben
\end{itemize}

\subsection{Diplomarbeit fertigstellen}
Zeitraum 31.05.2023 – 01.09.2023\\\\
Tätigkeiten:
\begin{itemize}
    \item Fertigstellung der Dokumentation
\end{itemize}


\pagebreak

\section{Analyse des Umfelds / Projektplanung}
\begin{tabular}{|m{0.7\textwidth}|m{0.3\textwidth}|}
\hline
\multicolumn{2}{|c|}{\textbf{Analyse des Umfelds / Projektplanung von 01.09.2022 bis 30.11.2022}} \\
\hline
Tätigkeit & Hauptverantwortliche/r \\
\hline
Recherche der Programme & Schachinger, Usta \\
\hline
Recherche der Spiel-Engines & Schachinger \\
\hline
Umfeldanalyse Unity gegen Unreal Engine & Schachinger \\
\hline
Umfeldanalyse der unterschiedlichen Modellierprogramme & Usta \\
\hline
Kategorisierung der Projektschwerpunkte & Schachinger, Usta \\
\hline
Zeiteinteilung der Projektschwerpunkte & Schachinger, Usta \\
\hline
Einrichtung des Projektmanagementtools & Usta \\
\hline
\end{tabular}

\section{Erstellung des Grundgerüsts für den Prototyps}
\begin{tabular}{|m{0.7\textwidth}|m{0.3\textwidth}|}
\hline
\multicolumn{2}{|c|}{\textbf{Erstellung des Grundgerüsts für den Prototyps von 30.11.2022 bis 18.12.2022}} \\
\hline
Tätigkeit & Hauptverantwortliche/r \\
\hline
Erstellung der Low-Poly Assets & Usta \\
\hline
Design und Aufbau des ersten Levels & Usta \\
\hline
Implementierung der Charaktersteuerung & Usta \\
\hline
Implementierung der beweglichen Plattformen & Schachinger \\
\hline
Design und Integration der ersten Spielfiguren & Usta \\
\hline
\end{tabular}

\section{Fertigstellung des UI}
\begin{tabular}{|m{0.7\textwidth}|m{0.3\textwidth}|}
\hline
\multicolumn{2}{|c|}{\textbf{Fertigstellung des UI von 18.12.2022 bis 28.02.2023}} \\
\hline
Tätigkeit & Hauptverantwortliche/r \\
\hline
Planung des UI für eine optimale User Experience & Schachinger, Usta \\
\hline
Aufbau und Erstellung einer Menüführung & Schachinger \\
\hline
Erstellung einer Statistikoberfläche & Schachinger \\
\hline
\end{tabular}

\section{Fertigstellen des Prototyps}
\begin{tabular}{|m{0.7\textwidth}|m{0.3\textwidth}|}
\hline
\multicolumn{2}{|c|}{\textbf{Fertigstellung des Prototyps von 28.02.2023 bis 31.05.2023}} \\
\hline
Tätigkeit & Hauptverantwortliche/r \\
\hline
Bugs fixen & Schachinger, Usta \\
\hline
Implementierung der High-Poly Assets & Usta \\
\hline
Fertigstellung der Designaufgaben & Usta \\
\hline
Fertigstellung der Programmieraufgaben & Schachinger \\
\hline
\end{tabular}

\section{Diplomarbeit fertigstellen}
\begin{tabular}{|m{0.7\textwidth}|m{0.3\textwidth}|}
\hline
\multicolumn{2}{|c|}{\textbf{Diplomarbeit fertigstellen von 31.05.2023 bis 01.09.2023}} \\
\hline
Tätigkeit & Hauptverantwortliche/r \\
\hline
Fertigstellung der Dokumentation & Schachinger, Usta\\
\hline
\end{tabular}

\pagebreak

\section{Arbeitsplan}
\subsection{Planung}
\subsubsection{Pflichtenheft}
\begin{tabular}{|m{0.3\textwidth}|m{0.7\textwidth}|}
\hline
\textbf{Name} & \textbf{Tätigkeit / Verantwortung} \\
\hline
Schachinger Lukas & Autor, hauptverantwortlich \\
\hline
Usta Martin & Autor \\
\hline
\end{tabular}

\subsubsection{Systemspezifikation}
\begin{tabular}{|m{0.3\textwidth}|m{0.7\textwidth}|}
\hline
\textbf{Name} & \textbf{Tätigkeit / Verantwortung} \\
\hline
Schachinger Lukas & Autor, hauptverantwortlich \\
\hline
Usta Martin & Autor \\
\hline
\end{tabular}

\subsubsection{Projekt – und Arbeitsplan}
\begin{tabular}{|m{0.3\textwidth}|m{0.7\textwidth}|}
\hline
\textbf{Name} & \textbf{Tätigkeit / Verantwortung} \\
\hline
Schachinger Lukas & Autor, hauptverantwortlich \\
\hline
Usta Martin & Autor \\
\hline
\end{tabular}