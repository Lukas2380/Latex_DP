\pagebreak
\setauthorname{Lukas Schachinger}

\chapter{Fazit}
Das letzte Kapitel dieser Diplomarbeit widmet sich der Frage: \glqq \textbf{Wie entwickelt man ein Computerspiel von Grund auf} \grqq. Während der Arbeit haben wir uns umfassend damit befasst, worauf es bei der Spielentwicklung ankommt. \\

Angefangen hat es mit der Überlegung des grundlegenden Konzeptes des Spiels. Zudem befassten wir uns mit dem Game Designs, der Storytelling-Elemente und der Spielmechaniken. Diese Teile sind das Grundgerüst jedes Spiels und legen den Grundstein für die gesamte Entwicklung.\\

Die technische Umsetzung ist der nächstwichtigste Punkt bei der Kreation eines Spieles. Zuerst muss man sich mit der Auswahl der richtigen Game Engine und Grafikdesigntools auseinandersetzen. Erst dann kann man sich mit der eigentlichen Entwicklung beschäftigen.\\

Wie in dem Kapitel \verb+Entwicklung des Prototyps+ beschrieben kann es passieren, dass mehere Versionen des Spieles erstellt werden muss. Dies ist ein großer Teil bei der Entwicklung eines Spieles. Diese Angehensweise ist hilfreich bei dem Testen oder der Verfeinerung der verschiedenen Aspekte des Spieles. In unserem Beispiel haben wir mehrere Versionen des ersten Levels erstellt um kleine Verbesserungen einzubauen oder bekannte Fehler auszubessern.\\

Zusammenfassend lässt sich sagen, dass die Spielentwicklung sowohl kreativ als auch technisch anspruchsvoll ist. Mit Hingabe und der Bereitschaft zum Lernen können Entwickler wie wir beeindruckende Spielerlebnisse erschaffen. Die sich stetig verändernde Branche eröffnet zudem Möglichkeiten, eigene Visionen zu verwirklichen und sich mit den neuesten Techniken zu beschäftigen.