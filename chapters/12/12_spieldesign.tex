\pagebreak
\setauthorname{Martin Usta}
\chapter{Spieledesign}
\section{Spieledesign}

Das Spieldesign umfasst mehrere Gebiete der Spielentwicklung. Die Gebiete des Spieldesign sind:

\begin{itemize}
    \item Level Design - Umgebungsdesign 
    \item Theme Auswahl 
    \item Spiel Schwierigkeit 
    \item Music 
\end{itemize}

Bei diesen Gebieten der Spielentwicklung ist es wichtig, dass sie einstimmig sind und zueinander passen. Falls die Gebiete des Spiels-design nicht zueinander passen, so wird das Spiele-Erlebnis für den Endkonsumenten schlecht. 

\subsection{Leveldesign - Umgebungsdesign}
Das Leveldesign besteht aus zwei voneinander Abhängigen Gebieten. Das betrifft das Level und die Umgebung die es betrifft. Dabei geht es beim Leveldesign um: Spielfiguren, Plattformen, Levelaufbau, Levelphysik etc. 
Das Umgebungsdesign befasst sich mit der Umgebung des Spiels wie zum Beispiel: Himmel(Skybox), Berge, Flüssigkeiten, Sonneneinstrahlung, etc. 
Beide Aspekte entscheiden wie, realistisch das Spiel ausschaut. Es ist nicht zum Empfehlen den Grafikstyle zwischen den Properties und der Umgebung zu mischen. 

\subsection{Theme Auswahl}
Das Theme bestimmt welche Atmosphäre das Spiel haben soll. Bei dem Theme ist es wichtig das es mit dem Gameplay abgestimmt ist. Es gibt viele Arten von Themes von Horror bis zum Abenteuer ist alles dabei. Nach der Theme Auswahl kann die Planung für die  Game Assets beginnen. Zudem entscheidet das Theme weitere Spieldesign Aspekte: wie die Musik, die Spielmechaniken und die  Story. Wir haben uns für ein Fantasie Theme entschieden. Das heißt das die Grafik sehr bunt gestaltet ist und einen nicht realistischen Grafikstil. Bei einen Fantasie Theme ist die Auswahl der Gestaltung des Levels sehr umfangreich. 

\subsection{Spiel Schwierigkeit}

Bei der Spieleentwicklung ist es wichtig die Schwierigkeit des Spieles zu bestimmen. Schwierigkeit bestimmt den Flow des Spielverlaufs. Die Schwierigkeit sollte je nach Zielgruppe abgestimmt werden. Für die Spieler ist es wichtig, dass das man gefordert wird. Das gibt den User das Gefühl etwas erreicht zu haben. Zudem sollte das Spiel nicht zu schwer sein, damit der Spieler nicht frustriert wird. 


%https://www.nuclino.com/articles/level-design#:~:text=What%20is%20level%20design%3F,player%20and%20keep%20them%20engaged.
