\pagebreak
\setauthorname{Martin Usta}
\chapter{Spieledesign} % todo: maybe fix überschrifteneinheitlichkeit
\section{Spieledesign} % todo: spieledesign/spieldesign 

Das Spieldesign umfasst mehrere Gebiete der Spielentwicklung. Die verschiedenen Gebiete des Spieldesign sind:

\begin{itemize} % todo: überschriften überarbeiten?
    \item Level Design - Umgebungsdesign 
    \item Theme Auswahl 
    \item Spiel Schwierigkeit 
    \item Musik
\end{itemize}

Bei diesen Gebieten der Spielentwicklung ist es wichtig, dass sie das selbe Theme verfolgen und zueinander passen. Falls die Gebiete des \hl{Spiels-design} nicht zueinander passen, so wird das \hl{Spiele-Erlebnis} für den \hl{Endkonsumenten schlecht}. 

\subsection{Leveldesign - Umgebungsdesign}
Das Leveldesign besteht aus zwei voneinander abhängigen Gebieten. \hl{Das betrifft das Level und die Umgebung die es betrifft.} Dabei geht es beim Leveldesign um: Spielfiguren, Plattformen, Levelaufbau, Levelphysik etc. 
Das Umgebungsdesign befasst sich mit der Umgebung des Spiels wie zum Beispiel: Himmel(Skybox), Berge, Flüssigkeiten, Sonneneinstrahlung, etc. 
Beide Aspekte entscheiden wie realistisch das Spiel auf den Spieler wirkt. Es ist nicht zum Empfehlen den Grafikstyle zwischen den Properties und der Umgebung zu mischen.

Ein Level besteht immer aus statischen Objekte und dynamischen Objekte. %todo: maybe insert citation?

\pagebreak

\subsection{Statische Objekte} %todo: footnote einheitlich machen
Statische Objekte sind Spielelemente die keine Interaktionsmöglichkeit für den Spieler bieten. Sie dienen meisten als Prop\footnote[1]{ist ein Spielobjekt} welches als nicht bewegliches Levelobjekt zählt. \\\\
Ein statisches Objekt könnte zum Beispiel ein Bücherregal in einem Level sein. Dieses bietet dem Spieler keine Interaktionsmöglichkeiten sondern dient nur der Ästhetik des Raumes. \\

\hl{
    Ein weiteres Beispiel wären Häuser die im Spiel implementiert sind, diese sind meist als ortgebundenes Objekt in der Welt, welches nur den Sinn hat, als Location für das nexte Spielevent zu dienen. Auch Objekte die nur den Sinn haben die Welt lebhafter zu gestalten wie Bäume, Fässer, Steine usw. dienen nur also Dekoelement. Diese dienen nur als optische Bereicherung für das Level. 
}

\subsection{Dynamische Objekte}
Dynamische Objekte haben im Gegensatz zu statischen Objekten eine \hl{Programmierung || maybe passt 'Funktion' besser} im Hintergrund. Diese Objekte haben einen aktiven Einfluss auf das Spielgeschehen. Zudem können sie mit dem spielbaren Charakter interagieren. Hier muss aber wieder differenziert werden zwischen Umgebungsobjekt und NPC(non playable character).\\\\ % todo: footnote for npc + bibliotek eintrag
Ein Umgebungsobjekt wäre zum Beispiel eine Platform die sich bewegt. Diese hat einen aktiven Einfluss auf den Spieler. Meistens werden dynamische Umgebungsobjekte als Herausforderung dem Spieler entgegengestellt. Ein Beispiel wäre eine Falle, die den Spieler blockiert bis dieser einen versteckten Schalter findet. \hl{
    Die Anwendung für Umgebungsobjekte sind grenzenlos. %todo: maybe useless sentence
}\\\\
Ein NPC(non playable character) zählt zwar auch als dynamische Objekt aber hat einen anderen Nutzen als das Umgebungsobjekt. \hl{Der NPC kann viele Rollen haben. || Ein NPC muss nicht ein Gegner sein...} Zudem kann dieser den Spielbaren Charakter helfen oder auch ihn hindern sein Ziel zu erreichen. Eine Rolle wäre die eines Questgebers, dieser kann den Spieler ein Aufgabe geben, die dieser erfüllen muss um weiter im Spiel voranzuschreiten. \hl{Eine andere Rolle wäre} das der Spieler von einem NPC blockiert werden kann oder sogar den Spieler angreift um ihn zu besiegen.

\pagebreak

\section{Die Spielewelt}
In einem Spiel gibt es immer eine Spielewelt wo die Geschichte erzählt wird. Diese kann aus mehreren Level bestehen oder nur einem großen Level welches auch Openworld genannt wird.


\subsection {Sektionale Level}
Bei einen sektionalen Level besteht aus mehreren Level und Szenen. Der große Vorteil ist das Assets in einem Neuen Level wiederverwendet werden. weiteres ist,dass die Level unaphängig von einander sind. Somit können größere Storysprünge gemacht werden. Der Nachteil dabei ist, dass der Storyfluss unter anderem gestört wird falls die Sprünge zwischen den Level zu groß sind. 

\subsubsection{Spielobjekte in sektionalen Leveln}
In eine sektionalen Level sind die Spielobjekte nicht an die anderen Level gebunden. Somit können die Assets je nach Level komplett unterschiedlich sein. Nehmen wir das beispiel von dem Klassiker Super Mario 64. Hierbei ist ein Level sommerlich gestaltet und das nächste im Winterdesign. Zudem können dann im weiteren Spielverlauf Spielobjekte wiederverwendet werden. Die wiederverwendet Objekte werden nur wenig verändert, damit diese zum Leveltheme passen. Was auch Interessannt ist, die Tatsache auch Gegenerische Spielfiguren nur leicht verändert werden um deren Stärke zu Demostrieren. Zum Beispiel der Videospiel Klassiker Chicken Invaders. Bei diesem Spiel habe die Gegnerischen Spielfiguren das gleiche Model nur die Farbe der tragenen T-Shirts verändert sich. Zum Beispiel haltet ein Huhn mit einen Roten T-Shirt weniger aus als Eins mit Blauen usw. Hier bei wird eindeutig nur die Farbe geändert und somit viel Zeit ersparrt für das Designen neuer Gegner.

\subsection{Kamera und Sichten}
In einer Spielewelt gibt es immer eine Kamera, die das Spielgeschehen den User auch anzeigt. Diese kann sowohl statisch auf eine Szene im Spiel plaziert sein, als auch auf die Spielfigur selbst. Zum Beispiel wäre es bei einem Schachspiel essentiell das sich die Kamera nicht bewegt und die Linze auf das Spielbrett zeigt. Indesem fall währe die Sicht die Adlereperspektive auch Dropdown genannt. Hier zeigt die Kamera von Oben nach Unten.Dann gibt es noch unterschiede zwischen den sichten wie die Spielfigur gezeigt wird. Da wären folgende:

\begin{itemize}
    \item \textbf{First Person:}
    \noindent Die firtsperson Sicht zeigt die Sicht aus der Spielfigur
    \item \textbf{Third Person:}
    \noindent Die thirdperson Sicht zeigt sowohl die Spielfigur als auch alles runterum. 
\end{itemize}

\subsection{Theme Auswahl}
Das Theme bestimmt welche Atmosphäre das Spiel haben soll. Bei dem Theme ist es wichtig das es mit dem Gameplay abgestimmt ist. Es gibt viele Arten von Themes von Horror bis zum Abenteuer ist alles dabei. Nach der Theme Auswahl kann die Planung für die  Game Assets beginnen. Zudem entscheidet das Theme weitere Spieldesign Aspekte: wie die Musik, die Spielmechaniken und die  Story. Wir haben uns für ein Fantasie Theme entschieden. Das heißt das die Grafik sehr bunt gestaltet ist und einen nicht realistischen Grafikstil. Bei einen Fantasie Theme ist die Auswahl der Gestaltung des Levels sehr umfangreich. 


\subsection{Spiel Schwierigkeit}

Bei der Spieleentwicklung ist es wichtig die Schwierigkeit des Spieles zu bestimmen. Schwierigkeit bestimmt den Flow des Spielverlaufs. Die Schwierigkeit sollte je nach Zielgruppe abgestimmt werden. Für die Spieler ist es wichtig, dass das man gefordert wird. Das gibt den User das Gefühl etwas erreicht zu haben. Zudem sollte das Spiel nicht zu schwer sein, damit der Spieler nicht frustriert wird. 

%https://www.nuclino.com/articles/level-design#:~:text=What%20is%20level%20design%3F,player%20and%20keep%20them%20engaged.
