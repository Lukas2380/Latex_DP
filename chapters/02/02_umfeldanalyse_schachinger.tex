\chapter{Umfeldanalysen}
\section{Umfeldanalyse der Game Engines}
In der folgenden Umfeldanalyse werden drei verschiedene Game Engines miteinander verglichen. Anhand der Auswahlkriterien: \textit{Programmierung, Relevanz und Marktanteil} und \textit{Dokumentation und Lektüre} wird entschieden, welche Game Engine für die Diplomarbeit am besten geeignet ist.

\subsection{Allgemeine Auswahlkriterien}
\begin{itemize}
  \item \textbf{Relevanz und Marktanteil:}\\\\ In dem Vergleichspunkt Relevanz und Marktanteil wird gezeigt, für welche Arten von Spielen die Game Engine benutzt wurde. Es soll außerdem dargestellt werden, wie stark die Game Engine auf dem Markt vertreten ist.
  \item \textbf{Programmierung:}\\\\ Die Programmierung beschreibt, wie und mit welchen Programmiersprachen in der Game Engine umgegangen wird. Es wird unterschieden zwischen dem Schreiben von Spielskripten und dem visuellen Skripten.
  \item \textbf{Dokumentation und Lektüre:}\\\\ Der Vergleichspunkt Dokumentation und Lektüre beschreibt, wie viele und welche Art von Dokumentationen verfügbar sind. 
\end{itemize}

\pagebreak

\subsection{Unity}
\subsubsection{Relevanz und Marktanteil}
Unter allen Game Engines ist Unity die bekannteste. Es ist die meistgenutzte Game Engine für \bettergls{indie}{1}- und Handy-Spielentwickler. Unity unterstützt über 20 verschiedene Plattformen wie zum Beispiel iOS, Android, Windows oder Ähnlichem.

\subsubsection{Programmierung}
Bei Unity kann zwischen verschiedenen Code-Editoren gewählt werden, um mit C\# oder JavaScript zu programmieren und Skripte zu schreiben. Es gibt keine Möglichkeit visuell zu skripten.

\subsubsection{Dokumentation und Lektüre (Community)}
Unity bietet eine sehr gut beschriebene Online-Dokumentation. Zudem gibt es viele Tutorials auf YouTube und anderen Plattformen. Der Community-Support ist umfangreich und leicht zugänglich. Zusätzlich gibt es eine Vielzahl von Büchern und Kursen, die sich speziell mit Unity beschäftigen.

\pagebreak

\subsection{Unreal Engine}
\subsubsection{Relevanz und Marktanteil}
Unreal Engine ist nach Unity die zweitbekannteste Game Engine. Hauptsächlich werden PC- und Konsolenspiele mit der Unreal Engine entwickelt. Während sich Unity mehr auf 2D spezialisiert, liegt die Stärke von Unreal Engine in der Entwicklung von 3D-Spielen. In der AAA-Industrie liegt Unreal weit vor Unity. Spiele wie \verb+Fortnite+, \verb+Bioshock+, \verb+Sea of Thieves+, \verb+Star Wars: Jedi Fallen Order+ und viele weitere nutzen diese Engine.

\subsubsection{Programmierung}
Während in Unity hauptsächlich in C\# programmiert wird, ist die Programmiersprache von Unreal Engine C++. Eine andere Option für die Programmierung in dieser Engine ist das interne visuelle Skripten. Dieses nennt sich \textit{Blueprints} und ist eine einfache Alternative zum Programmieren. Die Logik hinter dem visuellen Skripten ist für Personen, die mit Programmieren nichts zu tun haben, leichter zu verstehen. Dadurch ermöglicht Unreal Engine viel mehr Menschen, eigene Spiele zu bauen.

\subsubsection{Dokumentation und Lektüre}
Unreal Engine bietet eine umfangreiche Online-Dokumentation. Es gibt eine große Auswahl an Tutorials, aber es kann schwieriger sein, diese zu erlernen, da C++ eine anspruchsvollere Programmiersprache ist als C\#. Es gibt auch viele Bücher und Kurse, die sich mit der Spielentwicklung mit Unreal Engine befassen. Epic Games, ein großer Unterstützer dieser Engine, hostet oft Live-Tutorials auf Twitch, bei denen viele Funktionen und Methoden vorgestellt werden.

\pagebreak

\subsection{Godot}
\subsubsection{Relevanz und Marktanteil}
Godot ist eine leistungsstarke Open-Source Game Engine, die vor allem für 2D-Spiele verwendet wird. Obwohl sie nicht so weit verbreitet ist wie Unity oder Unreal Engine, hat sie eine aktive und wachsende Community.

\subsubsection{Programmierung}
Godot verwendet seine eigene Skriptsprache namens GDScript, die einfach zu erlernen und zu verwenden ist. Zudem bietet diese Game Engine auch eine visuelle Skriptumgebung mit dem Godot Editor.

\subsubsection{Dokumentation und Lektüre}
Godot bietet eine umfangreiche Online-Dokumentation und viele Tutorials. Da es jedoch eine weniger bekannte Engine ist, kann es schwieriger sein, Unterstützung und Ressourcen zu finden als bei Unity oder Unreal Engine. Dennoch gibt es eine aktive Community, die hilfreiche Beiträge und Tutorials bereitstellt.

\pagebreak

\subsection{Entscheidung}
\subsubsection{Relevanz und Marktanteil}
Im Punkt Relevanz und Marktanteil liegen Unity und Unreal Engine eindeutig vorne. Der Hauptgrund dafür ist, dass Godot relativ neu auf dem Markt ist und noch nicht so etabliert ist wie die anderen beiden Game Engines.

\subsubsection{Programmierung}
Im Punkt Programmierung ging es uns darum, dass wir uns mit der Programmiersprache gut auskennen. Da uns C\# besser liegt als C++ und wir bereits Erfahrung mit Unity gemacht haben, liegt Unity in diesem Punkt eindeutig vorne.

\subsubsection{Dokumentation und Lektüre}
Alle drei Game Engines haben eine umfangreiche Online-Dokumentation und viele Tutorials. Es kann schwieriger sein, für Godot Online-Hilfe zu bestimmten Themen zu finden, da diese Game Engine weniger bekannt ist. Daher lag auch bei diesem Punkt die Entscheidung zwischen Unity und Unreal Engine.

\subsubsection{Die endgültige Entscheidung}
Die endgültige Entscheidung fiel auf Unity. Einerseits aufgrund unseres Vorwissens über diese Game Engine und die Programmiersprache. Andererseits aufgrund der ausführlichen Online-Dokumentation und der weit verbreiteten Community.

\pagebreak