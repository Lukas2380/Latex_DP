\pagebreak

\section{Umfeldanalyse Design Programm}
Bei dieser Umfeldanalyse geht es um die Vergleiche zwischen Blender, Maya und ZBrush. Dabei werden die technischen Eigenschaften der unterschiedlichen Softwareprodukte gegenübergestellt und verglichen. Das Ziel ist dabei, die bestmögliche Software für unsere Anforderungen zu finden.

\subsection{Auswahlkriterien}
Zudem ist es wichtig für unser Projekt möglichst schlichte und doch effiziente Lösungen zu haben. Die Entscheidung für das Programm wird in dieser Umfeldanalyse durch bestimmte Auswahlkriterien erleichtert. Die Auswahlkriterien, welche bei der Entscheidung helfen sind: 

\paragraph{Verfügbarkeit}
Bei der Verfügbarkeit wird erläutert, wie man eigentlich zu einem Modellierprogramm kommt. Was sind die Aufwände, um eine Lizenz für diese Programme zu bekommen. Andernfalls gibt es Programme, welche sogar kostenfrei sind.

\paragraph{Nutzen}
Der Nutzen ist dann gewährleistet, wenn das Programm die Funktionen bietet, welche benötigt werden, um eine 3D- Figur zu erstellen. Dabei ist es wichtig, dass viele Aufgabenbereiche für die 3D-Modeliereung abgedeckt werden.

\paragraph{Handhabung}
Die Handhabung ist dann zufriedenstellend, wenn der Endverbraucher wenig Zwischenschritte machen muss wie möglich. Zudem muss das Userinterface Benutzerfreundlich und leicht zu bedienen sein.

\paragraph{Dokumentation \& Lektüre}
Bei den Dokumentationen ist es wichtig, das viel davon online über dieses Produkt gibt. Dies kann helfen, um auftauchende Probleme zu beseitigen. Das kann sowohl eine online Dokumentation sein als auch ein Fachbuch über das bestimmte Produkt.

\pagebreak

\subsection{Umfeldanalyse: Blender}
Blender gibt es schon sehr lange und ist das Produkt, welches am häufigsten verwendet wird. Der Grund dafür ist, das Blender eine open-source Anwendung ist. Das bedeutet das man für Blender keine Lizenz braucht um zu Arbeiten. Außerdem ist die Spannweite von Tools und Features enorm.

\subsubsection{Auswahlkriterien}
\begin{itemize}
    \item Verfügbarkeit
    \item Nutzung
    \item Handhabung
    \item Dokumentation \& Lektüre
\end{itemize}

\paragraph{Verfügbarkeit}
Wie schon beschrieben ist Blender eine open-source Anwendung. Dies bezüglich fallen für den Endkonsumenten keine kosten für Blender. Außerdem kann jede Funktion von Blender verwenden werden ohne das eine Zahlung erforderlich ist.

\paragraph{Nutzung}
Blender hat eine weite Auswahl von Tools und Features. Dabei liegen die Schwerpunkte in Blender beim Modellieren und Animieren. Aber auch die Texturbearbeitung für das 3D-Modell ist möglich und sehr gut ausgereift.

\paragraph{Handhabung}
Die Handhabung ist ihn Blender am Anfang sehr überfordernd. Falls man aber schon länger mit Blender arbeitet sollte das auch kein Problem mehr sein.

\paragraph{Dokumentation \& Lektüre}
Gerade für Blender ist eine Dokumentation sehr wichtig. Dadurch das Blender ein open-source Programm ist, wird es auch von vielen Menschen verwendet. Das hilft sehr da dementsprechend viele Tutorials und Guidelines gibt, wo man sich entsprechend orientieren kann. Außerdem gibt es Bücher über Blender wovon die Meisten sehr nützlich sind.

\pagebreak

\subsection{Umfeldanalyse Maya}
Maya ist das Produkt, welches am häufigsten in der Industrie verwendet wird. Im Gegensatz zu Blender ist Maya keine open-source Anwendung. Dadurch ist man verpflichtet sich eine Maya Lizenz über finanzielle Mittel zu erwerben. Dieses Produkt läuft stabiler als Blender und ist deshalb in der Produktion sehr beliebt.

\subsubsection{Auswahlkriterien}
\begin{itemize}
    \item Verfügbarkeit
    \item Nutzung
    \item Handhabung
    \item Dokumentation \& Lektüre
\end{itemize}

\paragraph{Verfügbarkeit}
Maya kann man sich online auf der Autodesk Website erwerben. Gerade die Einzellizenz preislich sehr weit oben. Die Lizenz dauert ein Jahr und muss je nach Abo Modell erneuert werden. Aber man kann sich auch eine zeitgebundene Demoversion zum Testen holen.

\paragraph{Nutzung}
In diesem Programm kann man genauso wie in Blender modellieren und animieren. Maya ist zwar nicht so umfangreich wie Blender aber der Nutzen wird durch die Zahlreichen Funktionen trotzdem erfüllt. Durch das schlichte Design und die übersichtlichen Tools ist das Programm sehr beliebt bei den größeren Produzenten.

\paragraph{Handhabung}
Die Handhabung in dieser Anwendung ist sehr angenehm. Das Userinterface ist sehr gut angeordnet und extra für ein angenehmen Workflow designt worden. Außerdem muss man nicht wie bei anderen Produkten extra schritte machen damit es funktioniert.

\paragraph{Dokumentation \& Lektüre}
Es gibt für dieses Programm online Dokumentation und Bücher. Diese Dokumentationen sind sehr gut beschrieben und ist aktuell.

\pagebreak

\subsection{ZBrush}
Bei ZBrush handelt es sich ebenfalls um ein Produkt zum 3D-Modelieren. ZBrush wird von vielen Hobbyartisten verwendet. Auch in der Industrie findet ZBrush seine Verwendung. ZBrush verwendet eine eigene Technik, welche dazu führt, dass das 3D-Objekt leichter zu bearbeiten ist.

\subsubsection{Auswahlkriterien}
\begin{itemize}
    \item Verfügbarkeit
    \item Nutzung
    \item Handhabung
    \item Dokumentation \& Lektüre
\end{itemize}

\paragraph{Verfügbarkeit}
ZBrush kann auf deren Website erworben werden. Im Gegensatz zu Maya ist eine Einzellizenz von ZBrush günstig. Dieses Abo kann man jeden Monat kündigen. Zudem bietet ZBrush eine Demoversion an.

\paragraph{Nutzung}
Bei ZBrush wird der Fokus sehr auf das Modellieren gesetzt. ZBrush hat eine zahlreiche Ansammlung an Bearbeitungstools, mit dem das 3D-Modell zurecht geformt werden kann. Mit diesem Programm kann man nicht animieren.

\paragraph{Handhabung}
Die Bedienung dieses Programm ist nicht Anfänger freundlich. Die Anzahl der Tools kann für den Anfänger sehr überfordert sein. Weiteres unterscheidet sich das Layout im Gegensatz zu den anderen Programmen wie Blender oder Maya sehr.

\paragraph{Dokumentation \& Lektüre}
Es gibt für dieses Programm online Dokumentation und Bücher. Diese Dokumentationen sind sehr gut beschrieben und ist aktuell.

\pagebreak

\subsection{Analyse}
\subsubsection{Verfügbarkeit}
Im Punkt Verfügbarkeit kann man erkennen das Blender mit Abstand das bessere Produkt ist. Aufgrund das Blender eine Open-Source Anwendung ist, ist kein finanzieller Aufwand nötig.

\subsubsection{Nutzung}
Bei der Nutzung geht es uns darum ob alle nötigen Funktionen, die wir für die Spieleentwicklung brauchen auch vorhanden sind. Aus der Analyse kann man klar erkennen das, sowohl Blender als auch Mayer unsere Anforderungen bei weiten decken. Dadurch das ZBrush keine integrierte Animationsengine hat fällt das Programm weg. Dadurch das unseres Team mehr Erfahrung mit Blender hat, entscheiden wir uns auch bei diesem Punkt für Blender.

\subsubsection{Handhabung}
In der Handhabung haben die Programme ZBrush und Mayer besser abgeschnitten als Blender. Das schlichte Design von Mayer und die Workarounds mit Mayer sind wesentlich angenehmer. Zudem läuft Mayer stabiler als Blender.

\subsubsection{Dokumentation und Lektüre}
Da alle genannten Programme sehr häufig in der Industrie verwendet werden, ist es sehr schwierig zu sagen, welchen Programm die beste Dokumentation hat. Bei allen Programmen findet man, sowohl analog als auch in digitaler Form eine sehr umfangreiche Dokumentation.

\subsection{Die Entscheidung}
Dadurch das Blender eine freizugängliche Software ist, welche zusätzlich eine gute Dokumentation besitzt und die Erfahrungen die unser Team.
Die Entscheidung fiel auf Blender. Das Lag daran, das Blender eine freizugängliche Software ist. Zudem konnte unser Team schon umfangreiche Erfahrungen mit diesem Programm machen. Somit mussten wir uns nicht in einem neuen Programm einlernen.
