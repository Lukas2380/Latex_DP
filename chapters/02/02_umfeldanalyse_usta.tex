\setauthorname{Martin Usta}

\section{Umfeldanalyse Design Programm}
In dieser Umfeldanalyse geht es um die Vergleiche zwischen Blender, Maya und ZBrush. Dabei werden die technischen Eigenschaften der unterschiedlichen Softwareprodukte gegenübergestellt und verglichen. Das Ziel ist dabei, die bestmögliche Software für unsere Anforderungen zu finden.


\subsection{Allgemeine Auswahlkriterien}
Zudem ist es wichtig für unser Projekt möglichst schlichte und doch effiziente Lösungen zu haben. Die Entscheidung für das Programm wird in dieser Umfeldanalyse durch bestimmte Auswahlkriterien erleichtert. Die Auswahlkriterien, welche bei der Entscheidung helfen sind: 

\begin{itemize}
    \item \textbf{Verfügbarkeit} \\\\ Bei der Verfügbarkeit wird erläutert, wie man zu einem Modellierprogramm kommt. Was sind die Kosten, um eine Lizenz für diese Programme zu bekommen oder gibt es Programme, welche sogar kostenfrei sind.
    \item \textbf{Nutzen} \\\\Der Nutzen ist dann gewährleistet, wenn das Programm die Funktionen bietet, welche benötigt werden, um eine 3D-Figur zu erstellen. Dabei ist es wichtig, dass viele Funktionen, wie zum Beispiel Sculpen und Designen von 3D-Objekten zur Verfügung stehen.
    \item \textbf{Handhabung} \\\\Die Handhabung ist dann zufriedenstellend, wenn der Endverbraucher wenig Zwischenschritte machen muss wie möglich. Zudem muss das Userinterface Benutzerfreundlich und leicht zu bedienen sein.
    \item \textbf{Dokumentation und Lektüre} \\\\ Bei den Dokumentationen ist es wichtig, das viel davon online über dieses Produkt gibt. Dies kann helfen, um auftauchende Probleme zu beseitigen. Das kann sowohl eine online Dokumentation sein als auch ein Fachbuch über das bestimmte Produkt.
\end{itemize}




\pagebreak

\subsection{Blender}
Blender gibt es schon sehr lange und ist das Produkt, welches am häufigsten verwendet wird. Der Grund dafür ist, das Blender eine open-source Anwendung ist. Das bedeutet das man für Blender keine Lizenz braucht um zu Arbeiten. Außerdem ist die Spannweite von Tools und Features enorm.




\subsubsection{Verfügbarkeit}
 Dadurch das Blender Open Source ist, gibt es keine Kosten für den Anwender. Außerdem kann jede Funktion von Blender verwenden werden ohne das eine Zahlung erforderlich ist.

\subsubsection{Nutzung}
Blender hat eine große Auswahl von Tools und Features. Dabei liegen die Schwerpunkte in Blender beim Modellieren und Animieren. Aber auch die Texturbearbeitung für das 3D-Modell ist möglich und ausgereift.

\subsubsection{Handhabung}
Die Handhabung ist in Blender am Anfang sehr überfordernd. Zudem ist die Lernkurve sehr hoch. Aber dafür ist die Auswahl der Funktionalitäten sehr groß.

\subsubsection{Dokumentation und Lektüre}
Gerade für Blender ist eine Dokumentation sehr wichtig. Dadurch das Blender von vielen Menschen verwendet wird, ist die Dokumentation sehr umfangreich.Das hilft sehr, da dementsprechend viele Tutorials und Guidelines gibt. Außerdem gibt es Bücher über Blender wovon die meisten sehr nützlich sind.

\pagebreak

\subsection{Umfeldanalyse Maya}
Maya ist das Produkt, welches häufig in der Industrie verwendet wird. Maya ist keine Open Source Anwendung. Dadurch ist man verpflichtet sich eine Lizenz über finanzielle Mittel zu holen. Dieses Produkt läuft stabiler als Blender und ist deshalb in der Produktion sehr beliebt.


\subsubsection{Verfügbarkeit}
Maya kann man sich online auf der Autodesk Website erwerben. Gerade die Einzellizenz preislich sehr weit oben. Die Lizenz dauert ein Jahr und muss je nach Abo Modell erneuert werden. Es gibt eine zeitgebundene Demoversion zum Testen.

\subsubsection{Nutzung}
In diesem Programm kann man genauso wie in Blender modellieren und animieren. Maya ist zwar nicht so umfangreich wie Blender aber der Nutzen wird durch die zahlreichen Funktionen trotzdem erfüllt. Durch das schlichte Design und die übersichtlichen Tools ist das Programm sehr beliebt bei den größeren Produzenten.

\subsubsection{Handhabung}
Die Handhabung in dieser Anwendung ist sehr angenehm. Das Userinterface ist sehr gut angeordnet und extra für ein angenehmen Workflow entworfen worden.

\subsubsection{Dokumentation und Lektüre}
Es gibt für dieses Programm online Dokumentation und Bücher. Diese Dokumentationen sind sehr gut beschrieben und ist aktuell.

\pagebreak

\subsection{ZBrush}
Bei ZBrush handelt es sich ebenfalls um ein Produkt zum 3D-Modelieren. ZBrush wird von vielen Hobbyartisten verwendet. Auch in der Industrie findet ZBrush seine Verwendung. ZBrush verwendet eine eigene Technik, welche dazu führt, dass das 3D-Objekt leichter zu bearbeiten ist.

\subsubsection{Verfügbarkeit}
ZBrush kann auf der offiziellen Website erworben werden. Im Gegensatz zu Maya ist eine Einzellizenz von ZBrush günstig. Dieses Abo kann jedes Monat gekündigen werden. Zudem bietet ZBrush eine Demoversion an.

\subsubsection{Nutzung}
Bei ZBrush wird der Fokus sehr auf das Modellieren gesetzt. ZBrush hat eine große Auswahl an Bearbeitungstools, mit dem das 3D-Modell zurecht geformt werden kann. Es ist nicht möglich mit diesem Programm zu animieren.

\subsubsection{Handhabung}
Die Bedienung dieses Programmes ist nicht sehr Anfängerfreundlich. Die Anzahl der Tools kann für einen Anfänger sehr überfordernd sein. Weiteres unterscheidet sich das Layout im Gegensatz zu den anderen Programmen.

\subsubsection{Dokumentation und Lektüre}
Es gibt für dieses Programm online Dokumentation und Bücher. Diese Dokumentationen sind sehr gut beschrieben und aktuell.

\pagebreak

\subsection{Entscheidung}
\subsubsection{Verfügbarkeit}
Im Punkt Verfügbarkeit kann man erkennen das Blender mit Abstand das bessere Produkt ist. Aufgrund das Blender eine Open-Source Anwendung ist, ist kein finanzieller Aufwand nötig.

\subsubsection{Nutzung}
Bei der Nutzung geht es darum ob alle nötigen Funktionen, die wir für die Spieleentwicklung brauchen auch vorhanden sind. Aus der Analyse kann man klar erkennen das, sowohl Blender als auch Maya die Anforderungen erfüllen. Dadurch das ZBrush keine integrierte Animationsengine hat fällt das Programm weg. Unsere Entscheidung fiel auf Blender, da unser Team mehr Erfahrung mit diesem Programm hat. 

\subsubsection{Handhabung}
In der Handhabung haben die Programme ZBrush und Maya besser abgeschnitten als Blender. Das schlichte Design von Maya und die Workarounds  sind wesentlich angenehmer. Zudem läuft Maya stabiler als Blender.

\subsubsection{Dokumentation und Lektüre}
Da alle genannten Programme sehr häufig in der Industrie verwendet werden, ist es sehr schwierig zu sagen, welches Programm die beste Dokumentation hat. Bei allen Programmen findet man, sowohl analog als auch in digitaler Form eine sehr umfangreiche Dokumentation. Auch bei diesem Punkt fiel unsere Entscheidung auf Blender. 

\subsection{Die Entscheidung}
Die Entscheidung fiel auf Blender. Das Lag daran, das Blender eine freizugängliche Software ist. Zudem konnte unser Team schon umfangreiche Erfahrungen mit diesem Programm machen. Somit mussten wir uns nicht in einem neuen Programm einlernen.