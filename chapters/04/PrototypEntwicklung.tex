\pagebreak
\chapter{Entwicklung des Prototyps}

In diesem Kapitel werden alle Schritte, die in die Kreation des Spiels hineingeflossen sind, beschrieben. Die Entwicklung des Prototypen fing mit der Idee des Themes an und durchlief mehere Phasen bis zur finalen Version des Spiels. Bei jeder dieser Phasen wurden viele Optimierungen gemacht und Fehler ausgebessert. 


\section{Idee und Thema des Spiels}


\begin{description}
  \item [Einführung in die Ausgangsidee des Spiels]
    Wegen des Vorwissens über Unity, welches in der 1 \& 2.ten Klasse unterrichte wurde war die Ausgangsidee für das Spiel war eine einfache Wahl. Die fast einfachste und am weitesten verbreitete Kategorie von Spiel ist ein \gls{jumprun} \footnote[1]{Bedeutung: Spring \& Lauf. Ein Spiel in dem man mit Springen und Laufen ein Ziel erreichen muss.}. 

    \begin{quote}
      \emph{\glqq Ein Merkmal (von Jump'n'Run Spielen) sind die Plattformen. [...] Typisch beim Jump'n'Run ist das Verlieren\grqq}~\cite[1:42-1:57]{ArtOfGaming}
    \end{quote}
    
    Ein Jump'n'Run würde einerseits kreative Freiheit über das Theme geben, aber auch die Möglichkeit bieten viele Aspekte der Spielentwicklung in dieser Diplomarbeit zu präsentieren.

  \item [Beschreibung des gewählten Themas und dessen Bedeutung]
    Da das Spiel nicht an der realen Welt angelegt sein soll, kam die überlegung eine Traumwelt zu bauen. In dieser gäbe es die Freiheit den Charakter, die Gegner und alle anderen Assets in verschiedenen Stilen zu designen. Damit wird auch für die Spieler dieses Prototypen klar das das gewählte Theme Fantasy ist. Diese Themenwahl erleichtert zudem dem Game-Designer die kreative Gestaltung.

  %\item [Erläuterung der Inspirationsquellen und kreativen Einflüsse]
  
\end{description}




\pagebreak

