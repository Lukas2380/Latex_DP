\chapter{User Interface (UI) und Interaktion}

Es ist erlaubt fremde Quellen zu zitieren, solange dies eindeutig erkennbar ist und sich die zitierten Stellen auf einen kurzen Absatz mit wenigen Zeilen beschränken \parencite[vgl.][S. 10]{quelle1}. Auch wenn nicht wörtlich zitiert wird, ist die Quelle der Information anzugeben.

Wörtliche Zitate werden durch Anführungszeichen begonnen und beendet sowie kursiv geschrieben: \enquote{Wissenschaftliches Plagiat: Man kann sich zwar mit fremden Federn schmücken, aber man kann nicht mit ihnen fliegen} \parencite[S. 15]{quelle3}.

Längere Zitate werden durch Einrücken vom normalen Text abgesetzt:

\begin{quote}
\textit{We tend to think of navigating a website as clicking from page-to-page via some kind of global navigation that's always visible. When it comes to a single page, we often think scrolling is the one and only way to move from one end to the next.} \parencite[vgl.][S. 10]{rehfeld}
\end{quote}

Das UI im Spieldesign ist fast der wichtigste Teil der Spielentwicklung, wenn man das Design des eigentlichen Spieles außen vornimmt. 
Ohne eine gute Menu führung und ein gut designtes User Interface ist es schwer, dass Spielerinnen und Spieler das Spiel verstehen und spielen können. 
\linebreak

Das UI des Prototyps unterteilt sich in 3 verschiedene Aspekte:

\begin{itemize}
    \item Das Hauptmenu.
    \item Das User Interface während des Spieles.
    \item Das Pause Menu.
\end{itemize}

\noindent
Bei der Entscheidung, welche Komplexität das User Interface des Prototyps haben soll, fiel die Wahl auf ein simples Design. 
Das UI soll einerseits die Schlichtheit des eigentlichen Spieles wiederspiegeln und andererseits alle für das Spiel wichtigen Informationen darstellen.

\section{Gestaltung des Hauptmenüs}
\section{Game UI und Spielmechanik-Anzeigen}
\section{Optionen und Einstellungen}
\section{Pause-Funktion und Benutzerfreundlichkeit}

According to Rehfeld \cite{rehfeld}, game design involves various aspects...
