\newglossaryentryfast{multiplayer}{Multiplayer}{Beschreibt den Videospiel Modus bei dem mehrere Menschen mit oder gegeneinander spielen.}

\newglossaryentryfast{UI}{User Interface}{Beschreibt die Schnittstelle zwischen Mensch und Computer.}

\newglossaryentryfast{singleplayer}{Singleplayer}{Beschreibt den Videospielmodus, bei dem eine einzelne Person das Spiel alleine spielt.}

\newglossaryentryfast{theme}{Theme}{Beschreibt die Gestaltung, den Stil, die Handlung und die Atmosphäre des Spiels.}

\newglossaryentryfast{canvas}{Canvas}{Ein Canvas (Leinwand) ist eine Fläche in einer Grafik- oder Game Engine, auf der Grafiken, Texte oder Benutzeroberflächenlemente gerendert werden können.}

\newglossaryentryfast{gameObject}{Game-Objekt}{Ein Game-Objekt ist eine grundlegende Einheit in einer Game Engine, die verschiedene Komponenten und Eigenschaften haben kann und im Spiel platziert und manipuliert werden kann.}

\newglossaryentryfast{skybox}{Skybox}{Eine Skybox ist ein Grafikelement in einer 3D-Umgebung, das den Eindruck eines fernen Himmels oder einer weit entfernten Umgebung erzeugt.}

\newglossaryentryfast{code-behind}{Code-behind}{Code-behind bezieht sich auf den Programmcode, der hinter Logik oder Funktionalität liegt.}

\newglossaryentryfast{statemachine}{State Machine}{Eine State Machine (Zustandsmaschine) ist ein Konzept in der Softwareentwicklung, das zur Steuerung von Zuständen und Übergängen verwendet wird.}

\newglossaryentryfast{jumprun}{Jump'n'Run}{Jump'n'Run ist ein Videospielgenre, bei dem der Spieler mit Springen und Laufen versuchen muss Hindernisse zu überwinden und Gegner zu besiegen.}

\newglossaryentryfast{collider}{Collider}{Ein Collider ist eine Kollisionskomponente in einer Game Engine, die zur Erkennung von Kollisionen zwischen Game-Objekten verwendet wird.}

\newglossaryentryfast{deathzone}{Deathzone}{Eine Deathzone ist eine Todeszone in der Charakter sozusagen stirbt.}

\newglossaryentryfast{dachRegion}{D-A-CH Region}{Die D-A-CH Region besteht aus Deutschland, Österreich und der Schweiz.}

\newglossaryentryfast{SVoD}{SVod}{Subscription-Video-on-Demand, Video streaming für das Benutzer ein Abonnement abschließen.}

\newglossaryentryfast{indie}{Indie}{Indie bedeutet unabhänging. Ein Indie Spiel ist unabhängig von einem Herausgeber wie zum Beispiel \glqq Nintendo\grqq. }

\newglossaryentryfast{void}{Void}{Das Void ist die Dunkelheit unter der Spielwelt. Wenn ein Spieler hineinfällt stirbt dieser oder bekommnt Schaden.}

\newglossaryentryfast{index}{Index}{Index ist eine Stelle in einem Array.}

\newglossaryentryfast{iteriert}{Iteriert}{Iterieren ist der Prozess des durchlaufen einer Sammlung von Daten.}

\newglossaryentryfast{prop}{Prop}{Props interaktive Objekte sein, die vom Spieler genutzt werden können, oder rein ästhetische Elemente, die die Spielwelt bereichern.}

\newglossaryentryfast{npc}{NPC}{ein Non-Playable-Character ist eine vom Computer gesteuerte Spielfigur.}

\newglossaryentryfast{topology}{Topology}{Topology (Topologie) beschreibt in Computerspielen die räumliche Anordnung von Spielobjekten und deren Beziehungen zueinander.}

\newglossaryentryfast{hp}{HP}{HP steht für "Hit Points" (Trefferpunkte). In vielen Videospielen repräsentieren HP die Gesundheit oder Ausdauer eines Charakters.}

\newglossaryentryfast{gameplay}{Gameplay}{Gameplay ist ein entscheidender Aspekt für die Qualität und den Unterhaltungswert eines Spiels.}

\newglossaryentryfast{flow}{Flow}{Flow (Flow-Erlebnis) beschreibt einen Zustand des vertieften und mühelosen Spielens, bei dem der Spieler in einem Videospiel vollständig in die Aufgabe eintaucht.}

\newglossaryentryfast{gamemechaniken}{Gamemechaniken}{Gamemechaniken sind Spielelemente welche man benötigt um das Spiel zu verwenden zum Beispiel Steurung und Fähigkeiten.}

\newglossaryentryfast{platformer}{Platformer}{Ein Platformer (Plattformspiel) ist ein Spielgenre, bei dem die Spielfigur hauptsächlich auf Plattformen springt, klettert oder sich bewegt, um Hindernisse zu überwinden und das Spielziel zu erreichen.} %maybe Platformer ist ein Spielkonzept wo die Spielfigur auf Platformen spring um weiter zu kommen

\newglossaryentryfast{ide}{IDE}{Eine IDE ist eine integrierte Entwicklungsumgebung.}

\newglossaryentryfast{coroutine}{Coroutine}{Coroutinen werden häufig in der Programmierung verwendet, um asynchrone Aufgaben zu behandeln und die Leistung zu verbessern.}

\newglossaryentryfast{ARPU}{ARPU}{Kurz für: Average revenue per User, also der durchschnittliche Erlös pro Nutzer}

\newglossaryentryfast{AAA-Industrie}{AAA-Industrie}{Die AAA-Industrie repräsentiert den Sektor der Computerspielindustrie, der sich auf hochbudgetierte, qualitativ hochwertige Spiele fokussiert.}

\newglossaryentryfast{PNG}{PNG}{PNG ist ein gängiges Bildformat in der Computerspielindustrie, das verlustfreie Kompression für Grafiken und Bilder bietet.}