\documentclass[12pt, oneside, numbers=noenddot]{scrbook}  
\input{./includes/htl_definintions.tex}	

\usepackage[onehalfspacing]{setspace}
\renewcommand{\familydefault}{\sfdefault}

% Initialen der Authoren
\def\authorInitialsA{LS} % Lukas Schachinger
\def\authorInitialsB{UM} % Usta Martin

% Die Initialen werden verwendet um anzuzeigen wer welches Kapitel
% erstellt hat.

% Die Befehle \authorA - \authorB werden in den Kapitelüberschriften angefügt
\def\authorA{\textmd{\textsuperscript{\authorInitialsA}}}
\def\authorB{\textmd{\textsuperscript{\authorInitialsB}}}


% Titel der Arbeit:
\def\htlArbeitsthema{Unity Game Design und Development}	

\setlength{\headheight}{23.5071pt}
\addtolength{\topmargin}{-6.5071pt}

\begin{document}

	
\tableofcontents
\pagenumbering{arabic}	

\input{./chapters/03_contenta.tex}
\input{./chapters/04_contentb.tex}
\input{./chapters/05_contentc.tex}
\input{./chapters/06_evaluation.tex}
\input{./chapters/07_projectmanagement.tex}
\input{./chapters/08_futurework.tex}
\input{./chapters/09_relatedwork.tex}
\input{./chapters/10_conclusion.tex}


\appendix

\input{./chapters/99_acknowledgements.tex}

\addcontentsline{toc}{chapter}{Listings}
\lstlistoflistings

\addcontentsline{toc}{chapter}{List of Figures}
\listoffigures


\addcontentsline{toc}{chapter}{Bibliography}
\bibliographystyle{plain}
\bibliography{literature}

\input{./chapters/99_authors.tex}



\end{document}  