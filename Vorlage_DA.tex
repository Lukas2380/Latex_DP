\documentclass[german, 12pt, oneside, numbers=noenddot]{scrbook}
\input{./includes/htl_definintions.tex}	

\usepackage[onehalfspacing]{setspace}
\usepackage{csquotes}
\usepackage{graphicx}
\usepackage{uarial} % Load uarial package for Arial font

% C#
\usepackage{listings}

% Fußnotenpackage
\usepackage{fancyhdr}
\usepackage{lastpage}

% to_donotes
\usepackage{todonotes}

% highlighting
\usepackage{xcolor, soul}

% german
%\usepackage[ngerman]{babel}

% ende
\newcommand{\newglossaryentryfast}[2]{%
    \newglossaryentry{#1}{%
        name={#2},
        description={},
    }%
}

\usepackage[sort=use]{glossaries}
\makenoidxglossaries
\newglossaryentryfast{multiplayer}{Multiplayer}{Beschreibt den Videospiel Modus bei dem mehrere Menschen mit oder gegeneinander spielen}

\newglossaryentryfast{UI}{User Interface}{Beschreibt die Schnittstelle zwischen Mensch und Computer}

\newglossaryentryfast{singleplayer}{Singleplayer}{Beschreibt...}

\newglossaryentryfast{theme}{Theme}{Beschreibt...}

\newglossaryentryfast{canvas}{Canvas}{Beschreibung...}

\newglossaryentryfast{gameObject}{Game-Objekt}{Beschreibung...}

\newglossaryentryfast{skybox}{Skybox}{Beschreibt}

\newglossaryentryfast{code-behind}{Code-behind}{Ist...}

\newglossaryentryfast{statemachine}{State Machine}{...}

\newglossaryentryfast{jumprun}{Jump'n'Run}{Ist ...}

\newglossaryentryfast{boxcollider}{Box-Collider}{Ist...}

\newglossaryentryfast{deathzone}{Deathzone}{Eine Deathzone ist eine Todeszone in der Charakter sozusagen stirbt.}
%\glsaddallunused % if glossar is acting up

\usepackage[ natbib=true, style=numeric,sorting=none]{biblatex}
\addbibresource{media/literatur.bib}

% Initialen der Autoren
\def\authorInitialsA{LS} % Lukas Schachinger
\def\authorInitialsB{UM} % Usta Martin

% Die Initialen werden verwendet um anzuzeigen wer welches Kapitel
% erstellt hat.h

% Die Befehle \authorA - \authorB werden in den Kapitelüberschriften angefügt
\def\authorA{\textmd{\textsuperscript{\authorInitialsA}}}
\def\authorB{\textmd{\textsuperscript{\authorInitialsB}}}

% Titel der Arbeit:
\def\htlArbeitsthema{Unity Game Design und Development}	

%\setlength{\headheight}{23.5071pt}
\setlength{\headheight}{23.7771pt}
\addtolength{\topmargin}{-6.5071pt}

% noindent
\setlength\parindent{0pt}

\begin{document}
\setauthorname{}
\pagestyle{customfooter}


\pagenumbering{arabic}	
		
% !TEX root = ../Vorlage_DA.tex

%	########################################################
% 							Deckblatt
%	########################################################


\titlehead{%
\begin{tabular}{@{}lcr@{}}
\raisebox{-0.5\height}{\includegraphics[width=0.15\linewidth]{media/images/htl_moedling.png}} &
\begin{minipage}[c]{0.7\linewidth}
\begin{center}
{\bfseries\sffamily\large HÖHERE TECHNISCHE BUNDES - LEHR- UND\\[0.2em]
VERSUCHSANSTALT MÖDLING}\\[1ex]
{\normalsize Höhere Lehranstalt für Elektronik und Technische Informatik\\
Kolleg für Informatik}
{\textcolor{gray}{bzw. Aufbaulehrgang für Informatik}}
\end{center}
\end{minipage} &
\raisebox{-0.5\height}{\includegraphics[width=0.15\linewidth]{media/images/htl-bildung-mit-zukunft}}
\end{tabular} \\ \\
\noindent\rule{1\linewidth}{0.4pt}
}

\title{\vspace{4em}\htlArbeitsthema}
\subtitle{ {\Large Diplomarbeit}\\[1em]Schulautonomer Schwerpunkt\\ Programmieren und Software Engineering}

\author{\\[1em] 
\emph{ausgeführt im Schuljahr 2022/2023 von:} \\[1em] 
Lukas Schachinger, 6AAIFT\\[1ex] 
Martin Usta, 6AAIFT\\[2em]
\emph{Betreuer:} \\[1em]
 Dipl.-Ing. Niklas Hack
}
\date{\vspace{3\baselineskip}\today}

\begin{titlepage}	
\maketitle
\end{titlepage}

\input{./chapters/00/00_ee.tex}
\chapter{Dokumentation}

\renewcommand{\arraystretch}{2} % Anpassen der Zeilenhöhe

\begin{tabular}{|m{0.3\textwidth}|m{0.7\textwidth}|}
\hline
Namen der Verfasser/innen & Lukas Schachinger, Martin Usta \\
\hline
Jahrgang & 5AAIFT / 6AAIFT \\
\hline
Thema der Diplomarbeit & Das Thema der Diplomarbeit ist Unity Game Design und Development. Es wird mithilfe der Dokumentation und eines spielbaren Prototyps gezeigt, wie man ein Spiel entwickelt. 
In der Dokumentation werden die Vorgehensweisen und die Programme, welche verwendet werden beschrieben. \\
\hline
Kooperationspartner & / \\
\hline
\end{tabular}

\vspace{10pt}

\noindent
\begin{tabular}{|m{0.3\textwidth}|m{0.7\textwidth}|}
\hline
Aufgabenstellung & Teilaufgabe von Lukas Schachinger: \newline \newline Programmierung der physikalischen Grundfunktionen und Implementierung und Design einer grafischen Benutzeroberfläche. \newline \newline Teilaufgabe von Martin Usta: \newline \newline Aufbau der 3D Spielewelt und Design der verschiedenen Level. Programmierung und Design der aktiven Spielfiguren. Implementierung der Musik und Soundeffekte der Spielumgebung.\\
\hline
\end{tabular}

\pagebreak

\noindent
\begin{tabular}{|m{0.3\textwidth}|m{0.7\textwidth}|}
\hline
Realisierung & Die Spielobjekte werden mithilfe von Blender modelliert. \newline \newline Der Prototyp wird in Unity realisiert. Die fertigen Assets werden in Unity integriert und dann in das Spiel eingebaut. \newline \newline Die Skripte werden in C\# geschrieben und dann an die Spielobjekte gebunden. \\
\hline
\end{tabular}

\vspace{10pt}

\noindent
\begin{tabular}{|m{0.3\textwidth}|m{0.7\textwidth}|}
\hline
Ergebnisse & Das Ergebnis wird ein spielbarer Prototyp sowie eine umfassende Dokumentation zum Thema Spieleentwicklung sein. Es soll mithilfe der Dokumentation erkennbar sein wie der Prototyp geplant und entwickelt wurde.  \\
\hline
\end{tabular}

\pagebreak
\pagebreak
\noindent
\begin{tabular}{|m{0.3\textwidth}|m{0.7\textwidth}|}
\hline
Typische Grafik, Foto etc. (mit Erläuterung) &
\begin{minipage}{\linewidth}
  \centering
  \begin{tabular}{cccc}
    \includegraphics[width=0.2\textwidth]{chapters/00/images/latex.png} &
    \includegraphics[width=0.2\textwidth]{chapters/00/images/unity.png} &
    \includegraphics[width=0.2\textwidth]{chapters/00/images/Csharp.png} &
    \includegraphics[width=0.2\textwidth]{chapters/00/images/blender.png}
  \end{tabular}
  
  \vspace{10pt}
  
  \begin{flushleft}
    Diese Abbildungen veranschaulichen sämtliche Produkte, die in dieser Diplomarbeit verwendet wurden. Die Dokumentation wurde mit Latex erstellt, der Prototyp mit Unity angefertigt und in C\# programmiert. In Blender wurden die Level und Assets modelliert.
  \end{flushleft}
  
  \vspace{10pt}

\end{minipage} \\
\hline
\end{tabular}

\vspace{10pt}


\noindent
\begin{tabular}{|m{0.3\textwidth}|m{0.7\textwidth}|}
\hline
Teilnahme an Wettbewerben, Auszeichnungen & Keine. \\
\hline
\end{tabular}

\vspace{10pt}

\noindent
\begin{tabular}{|m{0.3\textwidth}|m{0.7\textwidth}|}
\hline
Möglichkeiten der Einsichtnahme in die Arbeit & Im Archiv der Abteilung Elektronik und Technische Informatik der HTL Mödling. \\
\hline
\end{tabular}

\vspace{10pt}

\noindent
\begin{tabular}{|m{0.325\textwidth}|m{0.325\textwidth}|m{0.325\textwidth}|}
\hline
Approbation \newline (Datum / Unterschrift) & 
{\tiny Prüfer/Prüferin} \newline \newline \vspace{30pt} & 
{\tiny Direktor/Direktorin} \newline {\tiny Abteilungsvorstand/Abteilungsvorständin} \newline \newline \vspace{30pt} \\
\hline
\end{tabular}

\pagebreak

%here english ----------------------------------------------------------------

\noindent
\begin{tabular}{|m{0.3\textwidth}|m{0.7\textwidth}|}
\hline
Authors & Lukas Schachinger, Martin Usta \\
\hline
Academic Year & 5AAIFT / 6AAIFT \\
\hline
Topic & The topic of the diploma thesis is Unity Game Design and Development. It demonstrates the process of developing a game through documentation and a playable prototype. The documentation describes the methodologies and programs used. \\
\hline
Collaboration Partners & / \\
\hline
\end{tabular}

\vspace{10pt}

\noindent
\begin{tabular}{|m{0.3\textwidth}|m{0.7\textwidth}|}
\hline
Assignment & Subtask by Lukas Schachinger: \newline \newline Programming of the physics functions and implementation and design of a graphical user interface \newline \newline Subtask by Martin Usta: \newline \newline Construction of the 3D game world and design of various levels. Programming and design of active game characters. Implementation of music and sound effects in the game environment.\\
\hline
\end{tabular}

\pagebreak

\noindent
\begin{tabular}{|m{0.3\textwidth}|m{0.7\textwidth}|}
\hline
Implementation & The game objects are designed and modeled using Blender. \newline \newline The prototype is implemented in Unity. The finished assets are integrated into Unity and then incorporated into the game. \newline \newline The scripts are written in C\# and then bound to the game objects. \\
\hline
\end{tabular}

\vspace{10pt}

\noindent
\begin{tabular}{|m{0.3\textwidth}|m{0.7\textwidth}|}
\hline
Results & The result is a playable prototype and a detailed documentation on game development. The documentation should demonstrate how the prototype was planned and developed.  \\
\hline
\end{tabular}

\pagebreak

\pagebreak
\noindent
\begin{tabular}{|m{0.3\textwidth}|m{0.7\textwidth}|}
\hline
Typical Graphics, Photos, etc. (with Explanation) &
\begin{minipage}{\linewidth}
  \centering
  \begin{tabular}{cccc}
    \includegraphics[width=0.2\textwidth]{chapters/00/images/latex.png} &
    \includegraphics[width=0.2\textwidth]{chapters/00/images/unity.png} &
    \includegraphics[width=0.2\textwidth]{chapters/00/images/Csharp.png} &
    \includegraphics[width=0.2\textwidth]{chapters/00/images/blender.png}
  \end{tabular}
  
  \vspace{10pt}
  
  \begin{flushleft}
    The following images illustrate all the products used in this thesis. The documentation was created using LaTeX, the prototype was developed using Unity and programmed in C\#. Level design and asset modeling were done in Blender.
  \end{flushleft}
  
  \vspace{10pt}
  
\end{minipage} \\
\hline
\end{tabular}


\vspace{10pt}

\noindent
\begin{tabular}{|m{0.3\textwidth}|m{0.7\textwidth}|}
\hline
Participation in Competitions, Awards & None \\
\hline
\end{tabular}

\vspace{10pt}

\noindent
\begin{tabular}{|m{0.3\textwidth}|m{0.7\textwidth}|}
\hline
Accessibility of \newline Diploma Thesis & Stored in the archive of the secondary technical college of Moedling, department of electronics and computer engineering \\
\hline
\end{tabular}

\vspace{10pt}

\noindent
\begin{tabular}{|m{0.325\textwidth}|m{0.325\textwidth}|m{0.325\textwidth}|}
\hline
Approval (Date / Signature) &
 {\tiny Examiner \newline \newline} \vspace{30pt} &
  {\tiny Head of College / Department \newline \newline} \vspace{30pt} \\
\hline
\end{tabular}

	
\tableofcontents

\pagebreak
\setauthorname{Lukas Schachinger \& Martin Usta}
\chapter{Pflichtenheft}
\section{Zielbestimmung}

Es soll eine spielbare Demo erstellt werden. Dabei wird dem Spieler ein Charakter zur Verfügung gestellt, mit dem er sich in einer Spielwelt bewegen kann. Diese Spielwelt besteht aus verschiedenen selbst entworfenen Leveln. Es soll eine benutzerfreundliche Benutzeroberfläche geben, die aktuelle Spielinformationen anzeigt.

Zusätzlich wird eine Dokumentation erstellt, in der beschrieben wird, wie der Prototyp entwickelt wurde und welche Programme dabei verwendet wurden.

\subsection{Musskriterien}
Die Musskriterien beschreiben die Punkte, welche definitiv ein teil des Spieles und der Dokumentation sein sollen. Die Musikprogramme wurden zwar außerhalb der Dokumentation evaluiert, aber entzifferten sich als weniger wichtig als die verbesserte künstliche Intelligenz von den Wunschkriterien.

\begin{itemize}
  \item Spiele-Demo
  \begin{itemize}
    \item Spieler-Charakter
    \item Spielewelt
    \item Level-Design
  \end{itemize}
  \item Evaluierung von Programmen zur Spieleentwicklung
  \begin{itemize}
    \item Design: Blender, Material Design (selbstgestaltet/Download aus dem Internet)
    \item Game Engine: Unity
    \item Musik: (Musik-Software)
  \end{itemize}
\end{itemize}

\pagebreak

\subsection{Wunschkriterien}
Die Wunschkriterien beschreiben Punkte, welche dem Prototypen hinzugefügt werden könnten aber nicht erforderlich sind. Ein Multiplayer wurde nicht eingebaut, aber die verbesserte künstliche Intelligenz schien eine sinnfolle addition zu dem Prototypen. 

\begin{itemize}
  \item Multiplayer
  \item Verbesserte künstliche Intelligenz (AI)
\end{itemize}

\subsection{Abgrenzungskriterien}
Die folgenden Punkte gelten als Abgrenzungskriterien. Das bedeutet, dass diese Punkte nicht teil der Arbeit sind und diese daher nicht behandelt oder gemacht werden.

\begin{itemize}
  \item Kein vollständiges Spiel
  \item Kein VR-Spiel
\end{itemize}

\section{Projektumfeldanalyse}
Diese Arbeit baut nicht auf bestehenden Projekten auf.

\chapter{Projektplan}
\section{Gesamtprojektplan}

\newcolumntype{C}[1]{>{\centering\arraybackslash}p{#1}}

\noindent
\begin{tabular}{|C{3cm}|C{3cm}|C{3cm}|C{3cm}|C{3cm}|}
\hline
\multicolumn{5}{|c|}{\cellcolor{green!20} Projekt} \\
\hline
\cellcolor{green!20}Analyse des Umfelds/Projektplanung & & & & \\
\hline
& \cellcolor{green!20} Erstellung des Grundgerüsts für den Prototyp & & & \\
\hline
& & \cellcolor{green!20} Fertigstellung der UI & & \\
\hline
& & & \cellcolor{green!20} Fertigstellung des Prototyps & \\
\hline
& & & & \cellcolor{green!20} Diplomarbeit Fertigstellen \\
\hline
\end{tabular}


\vspace{10pt}

\noindent
\begin{tabular}{|C{2.43cm}|C{2.43cm}|C{2.43cm}|C{2.43cm}|C{2.43cm}|C{2.43cm}|C{2.43cm}|}
    Projektstart & Meilenstein 1 Ende & Meilenstein 2 Ende & Meilenstein 3 Ende & Meilenstein 4 Ende & Meilenstein 5 Ende \\
    01.09.2022 & 30.11.2022 & 18.12.2022 & 28.02.2023 & 31.05.2023 & 01.09.2023 \\
\end{tabular}


\pagebreak

\section{Meilensteine}
\subsection{Analyse des Umfelds / Projektplanung}
Zeitraum: 01.09.2022 – 30.11.2022\\\\
Tätigkeiten:
\begin{itemize}
    \item Recherche der Programme
    \item Recherche der Spiel-Engines
    \item Umfeldanalyse Unity gegen Unreal Engine
    \item Umfeldanalyse der unterschiedlichen Modellierprogramme
    \item Kategorisierung der Projektschwerpunkte
    \item Zeiteinteilung der Projektschwerpunkte
    \item Einrichtung des Projektmanagementtools 
\end{itemize}

\subsection{Erstellung des Grundgerüsts für den Prototyps}
Zeitraum: 30.11.2022 – 18.12.2022\\\\
Tätigkeiten:
\begin{itemize}
    \item Erstellung der Low-Poly Assets
    \item Design und Aufbau des ersten Levels 
    \item Implementierung der Charaktersteuerung
    \item Implementierung der beweglichen Plattformen
    \item Design und Integration der ersten Spielfiguren
\end{itemize}
\pagebreak

\subsection{Fertigstellung des UI}
Zeitraum 18.12.2022 – 28.02.2023\\\\
Tätigkeiten:
\begin{itemize}
    \item Planung des UI für eine optimale User Experience
    \item Aufbau und Erstellung einer Menüführung
    \item Erstellung einer Statistikoberfläche
\end{itemize}

\subsection{Fertigstellen des Prototyps}
Zeitraum 28.02.2023 – 31.05.2023\\\\
Tätigkeiten:
\begin{itemize}
    \item Bugs fixen
    \item Implementierung der High-Poly Assets
    \item Fertigstellung der Designaufgaben
    \item Fertigstellung der Programmieraufgaben
\end{itemize}

\subsection{Diplomarbeit fertigstellen}
Zeitraum 31.05.2023 – 01.09.2023\\\\
Tätigkeiten:
\begin{itemize}
    \item Fertigstellung der Dokumentation
\end{itemize}


\pagebreak

\section{Analyse des Umfelds / Projektplanung}
\begin{tabular}{|m{0.7\textwidth}|m{0.3\textwidth}|}
\hline
\multicolumn{2}{|c|}{\textbf{Analyse des Umfelds / Projektplanung von 01.09.2022 bis 30.11.2022}} \\
\hline
Tätigkeit & Hauptverantwortliche/r \\
\hline
Recherche der Programme & Schachinger, Usta \\
\hline
Recherche der Spiel-Engines & Schachinger \\
\hline
Umfeldanalyse Unity gegen Unreal Engine & Schachinger \\
\hline
Umfeldanalyse der unterschiedlichen Modellierprogramme & Usta \\
\hline
Kategorisierung der Projektschwerpunkte & Schachinger, Usta \\
\hline
Zeiteinteilung der Projektschwerpunkte & Schachinger, Usta \\
\hline
Einrichtung des Projektmanagementtools & Usta \\
\hline
\end{tabular}

\section{Erstellung des Grundgerüsts für den Prototyps}
\begin{tabular}{|m{0.7\textwidth}|m{0.3\textwidth}|}
\hline
\multicolumn{2}{|c|}{\textbf{Erstellung des Grundgerüsts für den Prototyps von 30.11.2022 bis 18.12.2022}} \\
\hline
Tätigkeit & Hauptverantwortliche/r \\
\hline
Erstellung der Low-Poly Assets & Usta \\
\hline
Design und Aufbau des ersten Levels & Usta \\
\hline
Implementierung der Charaktersteuerung & Usta \\
\hline
Implementierung der beweglichen Plattformen & Schachinger \\
\hline
Design und Integration der ersten Spielfiguren & Usta \\
\hline
\end{tabular}

\section{Fertigstellung des UI}
\begin{tabular}{|m{0.7\textwidth}|m{0.3\textwidth}|}
\hline
\multicolumn{2}{|c|}{\textbf{Fertigstellung des UI von 18.12.2022 bis 28.02.2023}} \\
\hline
Tätigkeit & Hauptverantwortliche/r \\
\hline
Planung des UI für eine optimale User Experience & Schachinger, Usta \\
\hline
Aufbau und Erstellung einer Menüführung & Schachinger \\
\hline
Erstellung einer Statistikoberfläche & Schachinger \\
\hline
\end{tabular}

\section{Fertigstellen des Prototyps}
\begin{tabular}{|m{0.7\textwidth}|m{0.3\textwidth}|}
\hline
\multicolumn{2}{|c|}{\textbf{Fertigstellung des Prototyps von 28.02.2023 bis 31.05.2023}} \\
\hline
Tätigkeit & Hauptverantwortliche/r \\
\hline
Bugs fixen & Schachinger, Usta \\
\hline
Implementierung der High-Poly Assets & Usta \\
\hline
Fertigstellung der Designaufgaben & Usta \\
\hline
Fertigstellung der Programmieraufgaben & Schachinger \\
\hline
\end{tabular}

\section{Diplomarbeit fertigstellen}
\begin{tabular}{|m{0.7\textwidth}|m{0.3\textwidth}|}
\hline
\multicolumn{2}{|c|}{\textbf{Diplomarbeit fertigstellen von 31.05.2023 bis 01.09.2023}} \\
\hline
Tätigkeit & Hauptverantwortliche/r \\
\hline
Fertigstellung der Dokumentation & Schachinger, Usta\\
\hline
\end{tabular}

\pagebreak

\section{Arbeitsplan}
\subsection{Planung}
\subsubsection{Pflichtenheft}
\begin{tabular}{|m{0.3\textwidth}|m{0.7\textwidth}|}
\hline
\textbf{Name} & \textbf{Tätigkeit / Verantwortung} \\
\hline
Schachinger Lukas & Autor, hauptverantwortlich \\
\hline
Usta Martin & Autor \\
\hline
\end{tabular}

\subsubsection{Systemspezifikation}
\begin{tabular}{|m{0.3\textwidth}|m{0.7\textwidth}|}
\hline
\textbf{Name} & \textbf{Tätigkeit / Verantwortung} \\
\hline
Schachinger Lukas & Autor, hauptverantwortlich \\
\hline
Usta Martin & Autor \\
\hline
\end{tabular}

\subsubsection{Projekt – und Arbeitsplan}
\begin{tabular}{|m{0.3\textwidth}|m{0.7\textwidth}|}
\hline
\textbf{Name} & \textbf{Tätigkeit / Verantwortung} \\
\hline
Schachinger Lukas & Autor, hauptverantwortlich \\
\hline
Usta Martin & Autor \\
\hline
\end{tabular}

\chapter*{Kurzfassung}
\addcontentsline{toc}{chapter}{Kurzfassung}

Diese Arbeit bietet eine umfassende Einführung in die Spieleentwicklung und den Prozess des Aufbaus eines Computerspiels mithilfe der Unity-Engine. Ziel ist es, zu zeigen, wie ein Spiel von Grund auf entwickelt wird und dabei sowohl die Herausforderungen als auch die Erfolge während der Planungs- und Umsetzungsphase zu beschreiben und darzustellen. Dabei wird ein Beispielprojekt für Unity Game Design und Development erstellt, das eine Vielzahl von Spielmechaniken von einfachen bis hin zu komplexen Funktionen enthält. Die Dokumentation bietet eine detaillierte Beschreibung dieser Mechaniken und geht darüber hinaus auf wichtige Aspekte der Spieleentwicklung ein, einschließlich der wirtschaftlichen Begründung für die Entwicklung von Spielen.

\chapter*{Abstract}
\addcontentsline{toc}{chapter}{Abstract}

This diploma thesis is an introduction to game development. This documentation shows the process of developing a 3D game in the Unity Engine. The thesis will describe the creation of a computer game. It will highlight the potential challenges that might be faced when building a videogame and ways to overcome those. Furthermore, a prototype will be developed with the thesis using Unity, showing various functions and design elements which are described in the documentation. The document includes a detailed description of the prototype and explores important aspects of game development, including the economic reasons behind creating a computer game.
\setauthorname{Lukas Schachinger}
\chapter{Einleitung}

Die Einleitung dieser Diplomarbeit, befässt sich zuerst mit der Problemstellung und dann der wirtschaftlichen Begründung für die Entwicklung eines Computerspieles in dem Jahr 2023. Alle Zahlen und Statistiken wurden aus den zugehörigen Seiten von der Webseite Statistika \cite{statistika}.

\section{Problemstellung}
Haben Sie sich schon einmal gefragt, wie man ein Computerspiel von Grund auf entwickelt?
\\ In dieser Diplomarbeit erforschen wir den Prozess der Spieleentwicklung mit Hilfe der Unity-Engine und betrachten detailliert den Entwicklungsprozess eines Spiels. Von der Idee bis zur Fertigstellung des Spiels teilen wir unsere eigenen Erfahrungen und Herausforderungen während der Entwicklung. Dadurch erhalten Sie einen Einblick in die verschiedenen Aspekte und Entscheidungen, die bei der Erschaffung eines Computerspiels berücksichtigt werden müssen.

\pagebreak

\section{Wirtschaftliche Begründung für die Entwicklung eines Spiels}
Der Gaming-Markt ist ein Milliardengeschäft. Allein in der D-A-CH-Region wird im Jahr 2023 ein Umsatz von rund 11,10 Milliarden Euro erwartet. Die Gaming-Branche bietet also ein enormes wirtschaftliches Potenzial.

\subsection{Die zunehmende Digitalisierung des Marktes}
Die Digitalisierung hat den Vertrieb von Spielen revolutioniert. Der Verkauf physischer Kopien geht zurück, während der digitale Vertrieb immer wichtiger wird. Dies ermöglicht eine breitere Verbreitung der Spiele und eine schnellere Verbindung mit den Spielern in der D-A-CH-Region.

\subsection{Die hohe Nutzerzahl als Umsatztreiber}
Prognosen zufolge werden im Jahr 2027 rund 42,77 Millionen Nutzer im Markt Videospiele in der D-A-CH-Region aktiv sein. Diese große Nutzerbasis bietet ein enormes Potenzial für Umsatzgenerierung. In-Game-Käufe, Abonnementmodelle und Werbung sind zusätzliche Einnahmequellen, von denen eine solche Nutzerbasis profitieren kann.

\subsection{Technologische Fortschritte}
Dank fortschreitender Technologie sind immer anspruchsvollere Spielerlebnisse möglich. Technologien wie Virtual Reality (VR) und Augmented Reality (AR) eröffnen völlig neue Möglichkeiten für die Spieleentwicklung und schaffen einzigartige Spielerlebnisse, die die Nutzer in der D-A-CH-Region begeistern.

\subsection{Chancen in einem wachsenden Markt}
Die Entwicklung eines eigenen Spiels bietet nicht nur hohe Umsatzpotenziale, sondern auch die Chance, teil einer dynamischen und innovativen Industrie zu sein. Die Gaming-Branche bietet Raum für Kreativität und ermöglicht es Ihnen, Ihre eigene Visionen und Ideen in ein erfolgreiches und unterhaltsames Produkt umzusetzen.

\pagebreak

\section{Vergleich zwischen Gaming und Video-Streaming-Services in der D-A-CH-Region}

\subsection{Umsatz}

Der Gaming-Markt generiert in Bezug auf Umsatz größere Zahlen als der Markt für Video-Streaming-Services. Im Jahr 2023 wird der Umsatz im Markt Video-Streaming-Services (SVoD) voraussichtlich rund 4,28 Milliarden Euro betragen, während der Umsatz im Markt Videospiele etwa 11,10 Milliarden Euro erreichen wird.

\subsection{Umsatzwachstum}

Das jährliche Umsatzwachstum im Markt Video-Streaming-Services (SVoD) wird für das Jahr 2027 auf 9,65\%  prognostiziert, während das jährliche Umsatzwachstum im Markt Videospiele bei 7,42\% liegt. Beide Märkte verzeichnen solides Wachstum, wobei der Video-Streaming-Services-Markt ein etwas höheres Wachstum aufweist.

\subsection{Nutzerzahl}

Die Anzahl der Nutzer spielt eine Rolle bei der Betrachtung der beiden Märkte. Prognosen zufolge werden im Jahr 2027 etwa 49,94 Millionen Nutzer Video-Streaming-Services (SVoD) nutzen, während im Markt Videospiele 42,77 Millionen Nutzer erwartet werden. Der Video-Streaming-Services-Markt hat eine größere Nutzerbasis, aber der Unterschied ist nicht signifikant.

\subsection{Durchschnittlicher Erlös pro Nutzer (ARPU)}

Der durchschnittliche Erlös pro Nutzer (ARPU) liegt im Jahr 2023 bei rund 105,70 Euro im Markt Video-Streaming-Services (SVoD) und bei 281,60 Euro im Markt Videospiele. Im Gaming-Markt können also pro Nutzer höhere Umsätze erzielt werden.

\section{Attraktive Chancen für die Entwicklung eines eigenen Spiels}

Trotz des Wachstums und Potenzials beider Märkte bietet die Gaming-Branche aufgrund ihrer höheren Umsätze und der Möglichkeit, ein breites Publikum anzusprechen, attraktive Chancen für die Entwicklung eines eigenen Spiels in der D-A-CH-Region.

\chapter{Umfeldanalysen}
\section{Umfeldanalyse der Game Engines}
In der folgenden Umfeldanalyse werden drei verschiedene Game Engines miteinander verglichen. Anhand der Auswahlkriterien: \textit{Programmierung, Relevanz und Marktanteil} und \textit{Dokumentation und Lektüre} wird entschieden, welche Game Engine für die Diplomarbeit am besten geeignet ist.

\subsection{Allgemeine Auswahlkriterien}
\begin{itemize}
  \item \textbf{Relevanz und Marktanteil:}\\\\ In dem Vergleichspunkt Relevanz und Marktanteil wird gezeigt, für welche Arten von Spielen die Game Engine benutzt wurde. Es soll außerdem dargestellt werden, wie stark die Game Engine auf dem Markt vertreten ist.
  \item \textbf{Programmierung:}\\\\ Die Programmierung beschreibt, wie und mit welchen Programmiersprachen in der Game Engine umgegangen wird. Es wird unterschieden zwischen dem Schreiben von Spielskripten und dem visuellen Skripten.
  \item \textbf{Dokumentation und Lektüre:}\\\\ Der Vergleichspunkt Dokumentation und Lektüre beschreibt, wie viele und welche Art von Dokumentationen verfügbar sind. 
\end{itemize}

\pagebreak

\subsection{Unity}
\subsubsection{Relevanz und Marktanteil}
Unter allen Game Engines ist Unity die bekannteste. Es ist die meistgenutzte Game Engine für \bettergls{indie}{1}- und Handy-Spielentwickler. Unity unterstützt über 20 verschiedene Plattformen wie zum Beispiel iOS, Android, Windows oder Ähnlichem.

\subsubsection{Programmierung}
Bei Unity kann zwischen verschiedenen Code-Editoren gewählt werden, um mit C\# oder JavaScript zu programmieren und Skripte zu schreiben. Es gibt keine Möglichkeit visuell zu skripten.

\subsubsection{Dokumentation und Lektüre (Community)}
Unity bietet eine sehr gut beschriebene Online-Dokumentation. Zudem gibt es viele Tutorials auf YouTube und anderen Plattformen. Der Community-Support ist umfangreich und leicht zugänglich. Zusätzlich gibt es eine Vielzahl von Büchern und Kursen, die sich speziell mit Unity beschäftigen.

\pagebreak

\subsection{Unreal Engine}
\subsubsection{Relevanz und Marktanteil}
Unreal Engine ist nach Unity die zweitbekannteste Game Engine. Hauptsächlich werden PC- und Konsolenspiele mit der Unreal Engine entwickelt. Während sich Unity mehr auf 2D spezialisiert, liegt die Stärke von Unreal Engine in der Entwicklung von 3D-Spielen. In der AAA-Industrie liegt Unreal weit vor Unity. Spiele wie \verb+Fortnite+, \verb+Bioshock+, \verb+Sea of Thieves+, \verb+Star Wars: Jedi Fallen Order+ und viele weitere nutzen diese Engine.

\subsubsection{Programmierung}
Während in Unity hauptsächlich in C\# programmiert wird, ist die Programmiersprache von Unreal Engine C++. Eine andere Option für die Programmierung in dieser Engine ist das interne visuelle Skripten. Dieses nennt sich \textit{Blueprints} und ist eine einfache Alternative zum Programmieren. Die Logik hinter dem visuellen Skripten ist für Personen, die mit Programmieren nichts zu tun haben, leichter zu verstehen. Dadurch ermöglicht Unreal Engine viel mehr Menschen, eigene Spiele zu bauen.

\subsubsection{Dokumentation und Lektüre}
Unreal Engine bietet eine umfangreiche Online-Dokumentation. Es gibt eine große Auswahl an Tutorials, aber es kann schwieriger sein, diese zu erlernen, da C++ eine anspruchsvollere Programmiersprache ist als C\#. Es gibt auch viele Bücher und Kurse, die sich mit der Spielentwicklung mit Unreal Engine befassen. Epic Games, ein großer Unterstützer dieser Engine, hostet oft Live-Tutorials auf Twitch, bei denen viele Funktionen und Methoden vorgestellt werden.

\pagebreak

\subsection{Godot}
\subsubsection{Relevanz und Marktanteil}
Godot ist eine leistungsstarke Open-Source Game Engine, die vor allem für 2D-Spiele verwendet wird. Obwohl sie nicht so weit verbreitet ist wie Unity oder Unreal Engine, hat sie eine aktive und wachsende Community.

\subsubsection{Programmierung}
Godot verwendet seine eigene Skriptsprache namens GDScript, die einfach zu erlernen und zu verwenden ist. Zudem bietet diese Game Engine auch eine visuelle Skriptumgebung mit dem Godot Editor.

\subsubsection{Dokumentation und Lektüre}
Godot bietet eine umfangreiche Online-Dokumentation und viele Tutorials. Da es jedoch eine weniger bekannte Engine ist, kann es schwieriger sein, Unterstützung und Ressourcen zu finden als bei Unity oder Unreal Engine. Dennoch gibt es eine aktive Community, die hilfreiche Beiträge und Tutorials bereitstellt.

\pagebreak

\subsection{Entscheidung}
\subsubsection{Relevanz und Marktanteil}
Im Punkt Relevanz und Marktanteil liegen Unity und Unreal Engine eindeutig vorne. Der Hauptgrund dafür ist, dass Godot relativ neu auf dem Markt ist und noch nicht so etabliert ist wie die anderen beiden Game Engines.

\subsubsection{Programmierung}
Im Punkt Programmierung ging es uns darum, dass wir uns mit der Programmiersprache gut auskennen. Da uns C\# besser liegt als C++ und wir bereits Erfahrung mit Unity gemacht haben, liegt Unity in diesem Punkt eindeutig vorne.

\subsubsection{Dokumentation und Lektüre}
Alle drei Game Engines haben eine umfangreiche Online-Dokumentation und viele Tutorials. Es kann schwieriger sein, für Godot Online-Hilfe zu bestimmten Themen zu finden, da diese Game Engine weniger bekannt ist. Daher lag auch bei diesem Punkt die Entscheidung zwischen Unity und Unreal Engine.

\subsubsection{Die endgültige Entscheidung}
Die endgültige Entscheidung fiel auf Unity. Einerseits aufgrund unseres Vorwissens über diese Game Engine und die Programmiersprache. Andererseits aufgrund der ausführlichen Online-Dokumentation und der weit verbreiteten Community.

\pagebreak
\setauthorname{Martin Usta}

\section{Umfeldanalyse Design Programm}
In dieser Umfeldanalyse geht es um die Vergleiche zwischen Blender, Maya und ZBrush. Dabei werden die technischen Eigenschaften der unterschiedlichen Softwareprodukte gegenübergestellt und verglichen. Das Ziel ist dabei, die bestmögliche Software für unsere Anforderungen zu finden.
\todo{pass die struktur an zu meiner umfeldanalyse}

\subsection{Allgemeine Auswahlkriterien}
Zudem ist es wichtig für unser Projekt möglichst schlichte und doch effiziente Lösungen zu haben. Die Entscheidung für das Programm wird in dieser Umfeldanalyse durch bestimmte Auswahlkriterien erleichtert. Die Auswahlkriterien, welche bei der Entscheidung helfen sind: 

\begin{itemize}
    \item \textbf{Verfügbarkeit} \\\\ Bei der Verfügbarkeit wird erläutert, wie man eigentlich zu einem Modellierprogramm kommt. Was sind die Aufwände, um eine Lizenz für diese Programme zu bekommen. Andernfalls gibt es Programme, welche sogar kostenfrei sind.
    \item \textbf{Nutzen} \\\\Der Nutzen ist dann gewährleistet, wenn das Programm die Funktionen bietet, welche benötigt werden, um eine 3D- Figur zu erstellen. Dabei ist es wichtig, dass viele Aufgabenbereiche für die 3D-Modeliereung abgedeckt werden.
    \item \textbf{Handhabung} \\\\Die Handhabung ist dann zufriedenstellend, wenn der Endverbraucher wenig Zwischenschritte machen muss wie möglich. Zudem muss das Userinterface Benutzerfreundlich und leicht zu bedienen sein.
    \item \textbf{Dokumentation \& Lektüre} \\\\ Bei den Dokumentationen ist es wichtig, das viel davon online über dieses Produkt gibt. Dies kann helfen, um auftauchende Probleme zu beseitigen. Das kann sowohl eine online Dokumentation sein als auch ein Fachbuch über das bestimmte Produkt.
\end{itemize}




\pagebreak

\subsection{Umfeldanalyse: Blender}
Blender gibt es schon sehr lange und ist das Produkt, welches am häufigsten verwendet wird. Der Grund dafür ist, das Blender eine open-source Anwendung ist. Das bedeutet das man für Blender keine Lizenz braucht um zu Arbeiten. Außerdem ist die Spannweite von Tools und Features enorm.




\subsubsection{Verfügbarkeit}
 Blender ist eine Open Source Anwendung. Dies bezüglich fallen für den Endkonsumenten keine kosten für Blender an. Außerdem kann jede Funktion von Blender verwenden werden ohne das eine Zahlung erforderlich ist.

\todo{open... schreibt man anders vl glossar eintrag}

\subsubsection{Nutzung}
Blender hat eine weite Auswahl von Tools und Features. Dabei liegen die Schwerpunkte in Blender beim Modellieren und Animieren. Aber auch die Texturbearbeitung für das 3D-Modell ist möglich und sehr gut ausgereift.

\subsubsection{Handhabung}
Die Handhabung ist in Blender am Anfang sehr überfordernd. Falls man aber schon länger mit Blender arbeitet sollte das auch kein Problem mehr sein.

\subsubsection{Dokumentation \& Lektüre}
Gerade für Blender ist eine Dokumentation sehr wichtig. Dadurch das Blender ein open-source Programm ist, wird es auch von vielen Menschen verwendet. Das hilft sehr da dementsprechend viele Tutorials und Guidelines gibt, wo man sich entsprechend orientieren kann. Außerdem gibt es Bücher über Blender wovon die meisten sehr nützlich sind.

\pagebreak

\subsection{Umfeldanalyse Maya}
Maya ist das Produkt, welches am häufigsten in der Industrie verwendet wird. Im Gegensatz zu Blender ist Maya keine Open Source Anwendung. Dadurch ist man verpflichtet sich eine hl  {Maya Lizenz | anders formulieren} über finanzielle Mittel zu erwerben. Dieses Produkt läuft stabiler als Blender und ist deshalb in der Produktion sehr beliebt.


\subsubsection{Verfügbarkeit}
Maya kann man sich online auf der Autodesk Website erwerben. Gerade die Einzellizenz preislich sehr weit oben. Die Lizenz dauert ein Jahr und muss je nach Abo Modell erneuert werden. Aber man kann sich auch eine zeitgebundene Demoversion zum Testen holen.

\subsubsection{Nutzung}
In diesem Programm kann man genauso wie in Blender modellieren und animieren. Maya ist zwar nicht so umfangreich wie Blender aber der Nutzen wird durch die zahlreichen Funktionen trotzdem erfüllt. Durch das schlichte Design und die übersichtlichen Tools ist das Programm sehr beliebt bei den größeren Produzenten.

\subsubsection{Handhabung}
Die Handhabung in dieser Anwendung ist sehr angenehm. Das Userinterface ist sehr gut angeordnet und extra für ein angenehmen Workflow entworfen worden. Außerdem muss man nicht wie bei anderen Produkten extra Schritte machen damit es funktioniert.

\subsubsection{Dokumentation \& Lektüre}
Es gibt für dieses Programm online Dokumentation und Bücher. Diese Dokumentationen sind sehr gut beschrieben und ist aktuell.

\pagebreak

\subsection{ZBrush}
Bei ZBrush handelt es sich ebenfalls um ein Produkt zum 3D-Modelieren. ZBrush wird von vielen Hobbyartisten verwendet. Auch in der Industrie findet ZBrush seine Verwendung. ZBrush verwendet eine eigene Technik, welche dazsubsubsectiondass das 3D-Objekt leichter zu bearbeiten ist.

\subsubsection{Auswahlkriterien}
\begin{itemize}
    \item Verfügbarkeit
    \item Nutzung
    \item Handhabung
    \item Dokumentation \& Lektüre
\end{itemize}

\subsubsection{Verfügbarkeit}
ZBrush kann auf ihrer Website erworben werden. Im Gegensatz zu Maya ist eine Einzellizenz von ZBrush günstig. Dieses Abo kann jedes Monat gekündigen werden. Zudem bietet ZBrush eine Demoversion an.

\subsubsection{Nutzung}
Bei ZBrush wird der Fokus sehr auf das Modellieren gesetzt. ZBrush hat eine zahlreiche Ansammlung an Bearbeitungstools, mit dem das 3D-Modell zurecht geformt werden kann. Mit diesem Programm ist es nicht möglich zu animieren.

\subsubsection{Handhabung}
Die Bedienung dieses Programm ist nicht seh Anfängerfreundlich. Die Anzahl der Tools kann für den Anfänger sehr überfordernd sein. Weiteres unterscheidet sich das Layout im Gegensatz zu den anderen Programmen wie Blender oder Maya sehr.

\subsubsection{Dokumentation \& Lektüre}
Es gibt für dieses Programm online Dokumentation und Bücher. Diese Dokumentationen sind sehr gut beschrieben und aktuell.

\pagebreak

\subsection{Entscheidung}
\subsubsection{Verfügbarkeit}
Im Punkt Verfügbarkeit kann man erkennen das Blender mit Abstand das bessere Produkt ist. Aufgrund das Blender eine Open-Source Anwendung ist, ist kein finanzieller Aufwand nötig.

\subsubsection{Nutzung}
Bei der Nutzung geht es uns darum ob alle nötigen Funktionen, die wir für die Spieleentwicklung brauchen auch vorhanden sind. Aus der Analyse kann man klar erkennen das, sowohl Blender als auch Maya unsere Anforderungen bei weiten decken.Dadurch das ZBrush keine integrierte Animationsengine hat fällt das Programm weg. Dadurch das unseres Team mehr Erfahrung mit Blender hat, entschieden wir uns auch bei diesem Punkt für Blender.

\subsubsection{Handhabung}
In der Handhabung haben die Programme ZBrush und Maya besser abgeschnitten als Blender. Das schlichte Design von Maya und die Workarounds  sind wesentlich angenehmer. Zudem läuft Maya stabiler als Blender.

\subsubsection{Dokumentation und Lektüre}
Da alle genannten Programme sehr häufig in der Industrie verwendet werden, ist es sehr schwierig zu sagen, welchen Programm die beste Dokumentation hat. Bei allen Programmen findet man, sowohl analog als auch in digitaler Form eine sehr umfangreiche Dokumentation.

\subsection{Die Entscheidung}
Die Entscheidung fiel auf Blender. Das Lag daran, das Blender eine freizugängliche Software ist. Zudem konnte unser Team schon umfangreiche Erfahrungen mit diesem Programm machen. Somit mussten wir uns nicht in einem neuen Programm einlernen.

%good till here;
\pagebreak
\setauthorname{Lukas Schachinger}
\chapter{Unity}

\section{Einführung}
Unity (auch bekannt als Unity3D) ist eine Game Engine und \bettergls{ide}{1} zum kreieren von Interaktionsmedien, meißt Video Spielen. \Cite[][A history of the unity game engine]{haas2014history} \\

Die Game Engine wurde in Kopenhagen 2005 veröffentlicht und ist bekannt für Spiele wie zum Beispiel: \verb+Among Us+, \verb+Monument Valley+, \verb+Pokemon Go+ oder dem Shooter \verb+Escape from Tarkov+. Die Game Engine wurde mit C++ geschrieben, aber ermöglicht den Nutzern mit der zugänglicheren Programmiersprache C\# zu arbeiten. Außerdem bietet die Game Engine die Möglichkeit mit einem Visuellen Editor die Level zu gestalten.
%todo: shooter gls?

\subsection{Game Objects und Komponenten}
Jedes Game-Object (Spiel-Objekt in Deutsch), kann eine Vielzahl von Komponenten annehmen. Die nächsten Unterkapitel beschreiben verschiedene solcher Komponenten und stellen diese bildlich dar.

\subsubsection{Das Mesh}
\glqq A mesh consists of triangles arranged in 3D space to create the impression of a solid object. \grqq \Cite[][Anatomy of a Mesh, Unity Documentation]{unitydoc}\\
Ein sogenanntes Mesh definiert das physikalische Objekt. Wie in diesem Unity Artikel beschrieben ist, besteht das Mesh eines Objektes aus, im 3 Dimensionalen Raum positionierten, Dreiecken welche ein solides Objekt darstellen sollen.\\\\
\noindent
\includegraphics[width=1\linewidth]{chapters/14/Images/Mesh2.png}

\pagebreak

\subsubsection{Der Mesh Renderer}
Der Mesh Renderer ist dafür verantwortlich Materalien und Lichteffekte wie Reflextionen bei einem Objekt darzustellen. Wie in der nächsten Abbildung dargestellt, kann mit dem Mesh Renderer ein Material dem Game-Objekt hinzugefügt werden.\\
\noindent

\begin{figure}[H]
  \centering
  \includegraphics[width=0.7\linewidth]{chapters/14/Images/MeshRenderer.png}
  \caption{Der Mesh Renderer als Komponent.}
  \label{U03}
\end{figure}

\subsubsection{Die Physics Komponenten} 
Wenn ein Rigidbody Komponent zu einem Objekt hinzugefügt wird, dann ist die Bewegung des Objektes unter der Kontrolle von Unitys Physik Engine. Selbst ohne Hinzufügen von Code wird ein Rigidbody-Objekt durch die Schwerkraft nach unten gezogen und auf Kollisionen mit einfallenden Objekten reagieren. Auch wenn das passende Collider-Komponente vorhanden ist. \cite[][Rigidbody, Unity Documentation] {unitydocRigidbody} \\ 

\noindent

\begin{figure}[H]
  \centering
  \includegraphics[width=0.7\linewidth]{chapters/14/Images/Physics.png}
  \caption{Die Komponenten für einen Box Collider und Rigidbody.}
  \label{U04}
\end{figure}

\pagebreak
\setauthorname{Lukas Schachinger \& Martin Usta}

\subsubsection{Skript Komponente}
Zudem ist es möglich einem Game-Objekt ein eigenes Skript als Komponente hinzuzufügen. Mit C\# Code ist es möglich eigene Fähigkeiten oder Attribute dem Objekt zu geben. Zum Beispiel kann eine Variable für \bettergls{hp}{1} angelegt werden. Wenn dieses Game-Objekt dann durch Kontakt mit einem Gegner Health Points abgezogen bekommt, werden Lebenspunkte abgezogen. Wenn diese dann auf Null gelangen, kann mit dem Code das Game-Objekt deaktiviert werden.


\subsection{Skybox}\
Eine Skybox ist der Hintergrund der Spielewelt. Dabei ist eine Skybox ein Würfel, indem sich die Spielewelt befindet. Die inneren Seiten des Würfels bilden den gesamten Hintergrund.
Zu beachten ist das die Würfelseiten die richtige Textur bekommen. Jede Seite hat seine eigene Koordinate für die vergebene Textur:\\\\

\begin{minipage}{0.4\textwidth}
    \begin{itemize}
        \item +Z für vorne 
        \item -Z für hinten 
        \item +X für links 
        \item -X für rechts
        \item +Y für oben
        \item -Y für unten
    \end{itemize}
  \end{minipage}
  \hfill
  \begin{minipage}{0.6\textwidth}
    \includegraphics[width=\linewidth]{chapters/14/Images/Skybox.png}
  \end{minipage}

\pagebreak
\subsection{Animation}

Beim Animieren geht es darum, dem Nutzer eine scheinbare Bewegung des Spielobjekts zu vermitteln. Dabei werden die beweglichen Abläufe eines Spielobjekts erstellt und aufgenommen. Hierbei werden die Koordinaten bestimmter Körperelemente in Keyframes festgehalten. Diese können dann auf einer Zeitachse platziert werden. Dadurch wird ein Keyframe mit den gespeicherten Koordinaten zum entsprechenden Zeitpunkt aufgerufen, um eine Bewegung zu simulieren. In dem Prototypen besitzt jede Münze eine Animationskomente.

\begin{figure}[H]
  \centering
  \includegraphics[width=0.5\linewidth]{chapters/14/Images/Animation.png}
  \caption{Die drehenden Münzen im ersten Level.}
  \label{U06}
\end{figure}

\subsection{Partikel des Portals}

Die Partikel des Portals sind mehrere Game-Objekte die den Partikel Komponenten haben. Ein Partikelsystem besteht aus einer Source und Partikeln. Dieser muss nur in der Größe und Rotation angepasst werden. Dem Partikel System müssen die Einstellungen und das Material angepasst werden um diesen Effekt zu bekommen.

\begin{figure}[H]
  \centering
  \includegraphics[width=0.5\linewidth]{chapters/14/Images/Partikel.png}
  \caption{Die Partikel des Portals.}
  \label{U07}
\end{figure}
\pagebreak
\setauthorname{Martin Usta}
\chapter{Blender}

\subsection{Einführung}
Blender ist ein umfangreiches Programm für die Erstellung von 3D-Objekten. Diese Software kann man auch zum Animieren, Zeichnen, Video Editieren oder Ähnlichem benutzen. 
In dieser Diplomarbeit wird der Fokus auf den 3D-Viewport und auf die Animation gesetzt. Das sind die wichtigsten Funktionen für die weiter Bearbeitung in Unity. 

\subsection{3D-Viewport}
Im 3D Viewport kann man das 3D-Objekt bearbeiten. Für diese Bearbeitung hat die Software folgende Tools: 

\begin{itemize}
    \item \textbf{Objekt Mode}:
    \indent Das Objekt als gesamtes kann bewegt und bearbeitet werden. 
    \item \textbf{Edit Mode}:
    \indent Einzelne Kanten, Punkte und Oberflächen können von einem Objekt verändert werden. 
    \item \textbf{Sculpting Mode}:
    \indent Bei diesem Mode kann das Objekt dynamisch verändert werden.
    \item \textbf{Vertex Paint}:
    \indent Einzelne Ecken, Punkte und Oberflächen können in eine separate Gruppe zusammen gegliedert werden
    \item \textbf{Weight Paint}: 
    \indent Weight Paint ist dafür da, um zu bestimmen, wie abhängig ein Objekt zu einen Animationsskelet ist. 
    \item \textbf{Texture Paint}:
    \indent Diese Funktion dient zur Bemalung des Objektes.
\end{itemize}

\pagebreak

\subsection{Objekt Mode}

Im Objekt Mode können die Objekte verändert werden.
Dabei kann man, die Größe, die Länge und die Breite geändert werden.
Außerdem kann das Objekt rotiert und die Position im 3-dimensionalen Raum verändert werden. 
Es können zudem weitere Objekte hinzugefügt werden.
\begin{figure}[H]
    \centering
    \includegraphics[width=0.8\textwidth]{chapters/13/images/3D-Viewport.png}
    \caption{Ein selectierter Würfel im Objekt Mode von Blender.}
    \label{UST-8}
\end{figure}


\subsection{Edit Mode}
Dieser Modus bietet eine Reihe von Tools, um die gewünschte Form des Objekts zu erzeugen.
Diese Tools sind:


\begin{itemize}
    \item \textbf{Extrudieren}:
    \indent Das Extrudieren wird verwendet , um eine Fläche höher oder tiefer zu ziehen und somit als neuer Würfel zu dem Objekt hinzuzufügen.
    \begin{figure}[H]
        \centering
        \includegraphics[width=0.8\textwidth]{chapters/13/images/ExtrudeTool.png}
        \caption{Eine Extration von einer Würfelfläche.}
        \label{UST-9}
    \end{figure}
    
    \item \textbf{Inset Faces}:
    \indent Das Inset Faces Tool ermöglicht das Erstellen von neuen Flächen innerhalb einer ausgewählten Fläche, was beispielsweise für die Erstellung von Löchern oder symmetrischen Auswuchtungen benötigt wird.
    \begin{figure}[H]
        \centering
        \includegraphics[width=0.8\textwidth]{chapters/13/images/InsetFace.png}
        \caption{Die Funktion Inset Faces erstellt eine neue Fläche auf einer ausgewählten Fläche.}
        \label{UST-10}
    \end{figure}
    
    \item \textbf{Loop Cut}:
    \indent Mit dem Loop Cut Tool können zusätzliche Kanten um das Objekt erstellt werden. Hiermit werden symmetrisch zum Objekt neue Kanten erstellt.
    \begin{figure}[H]
        \centering
        \includegraphics[width=0.8\textwidth]{chapters/13/images/Loopcut.png}
        \caption{Die Anwendung des Loopcut Tools wird auf einen Würfel verwendet.}
        \label{UST-11}
    \end{figure}
    \item \textbf{Messer}:
    \indent Das Messer Tool ermöglicht, Kanten mit der Freihand hinzuzufügen. Das wird benötigt, um Kanten an nicht symmetrischen Stellen zu erzeugen.
    \begin{figure}[H]
        \centering
        \includegraphics[width=0.8\textwidth]{chapters/13/images/KniveTool.png}
        \caption{Die Anwendung des Knive Tools wird auf einen Würfel verwendet.}
        \label{UST-12}
    \end{figure}
    \item \textbf{Poly Builder}: 
    \indent Schließlich bietet das Poly Builder Tool die Möglichkeit, einzelne Kanten zu markieren und die Länge zu verändern, um das Objekt nach Belieben zu verändern.
    \begin{figure}[H]
        \centering
        \includegraphics[width=0.8\textwidth]{chapters/13/images/PolyBuild.png}
        \caption{Die Anwendung des Poly Build Tools wird auf einen Würfel verwendet.}
        \label{UST-13}
    \end{figure}
\end{itemize}
%https://docs.blender.org/manual/en/latest/editors/3dview/modes.html

\subsection{Sculping Mode}
Der Sculping Mode hat verschiedene Tools um das Objekt dynamisch zu verändern. In diesem Tool kann die \bettergls{topology}{1} bearbeitet werden. Hier kann die Anzahl der Polygone wie im Edit Mode dynamisch verändert werden. Die Arbeitsweise in diesem Tool ist wie folgt.\\\\
Zuerst werden grobe Details des Objekt erzeugt wie zum Beispiel die Grundform des Objekt.\\\\
\begin{figure}[H]
    \centering
    \includegraphics[width=0.8\textwidth]{chapters/13/images/HolzBrett.png}
    \caption{Die Bearbeitung der Grundform eines Spielobjekts®.}
    \label{UST-14}
\end{figure}
\noindent Danach wird die Anzahl der Polygone erhöht um feinere Details hinzuzufügen. Dieser Workflow dient dazu so immer auf einer Detailstufe zu bleiben.
\begin{figure}[H]
    \centering
    \includegraphics[width=0.8\textwidth]{chapters/13/images/HolzBrett1.png}
    \caption{Die Bearbeitung gröberer Details mit einer höheren Polygonanzahl.}
    \label{UST-15}
\end{figure}
\noindent Dieser Ablauf wird 2 bis 3 mal wiederholt um ein schönes detailiertes Objekt zu erstellen.
\begin{figure}[H]
    \centering
    \includegraphics[width=0.8\textwidth]{chapters/13/images/HolzBrett2.png}
    \caption{Die Bearbeitung feinere Details auf das Objekt.}
    \label{UST-16}
\end{figure}
\noindent Wichtig ist darauf zu achten, dass je mehr Polygone das Objekt besitzt desto mehr muss die Cpu rechnen. Das kann zum weiteren, zur schlechteren Performance im Spiel verusachen. Der letzte Schritt ist dann wieder das Runterskalieren der Objekte. Dieser Schritt ist wichtig um unnötige Polygone nicht zu laden. Wichtig dabei ist, dass die Details bebehalten werden.Um das zu gewährleisten, werden Bereiche des Obejekt runter skaliert. In der Folgenden Abbildung kann man erkennen, dass die orange Fläche keine Details enthält und somit runterskaliert wurde.

\begin{figure}[H]
    \centering
    \includegraphics[width=0.8\textwidth]{chapters/13/images/HolzBrett3.png}
    \caption{Die runterskalierung einer bestimmten Fläche auf dem Objekt.}
    \label{UST-17}
\end{figure}


\pagebreak
\setauthorname{Martin Usta}
\chapter{Spieleperformance}



\subsection{Einführung}
In jedem Spiel ist es wichtig, dass das Spiel ohne Performanceverluste funktioniert. Die Performance wird von der Spielewelt und ihren Spielobjekten beeinträchtigt. Bei der Spieleentwicklung muss sehr oft abgewogen werden ob bestimmte Details ins Spiel implementiert sein müssen. Denn Jedes noch so kleine Detail braucht Rechenleistung. In der Spieleentwicklung werden nur selten Spielobjekte wegen des Leistungsverlustes entfernt. Meistens werden diese für das Spiel optimiert. Bei der Spieleentwicklung kann man die Performance anhand von zwei Faktoren optimieren. Ein Faktor betrifft dabei die Spiel Engine, während der andere Faktor das Modellierungsprogramm betrifft. 

\subsection{Spielobjekt Optimierung im Grafikprogramm}
Bei den Grafikprogrammen kann vieles gemacht werden um die Spielobjekt performance effizienter zu machen. Wenn ein Gameobjekt erstellt wird, besteht dieses aus Seiten und Flächen. Diese werden auch Polygone genannt. Die Polygonanzahl bestimmt den Detailgrad des Spielobjekts das heißt je mehr Polygone das Objekt besitzt desto detailreicher ist es. Wenn ein Objekt viele Polygone besitzt, kann es sein, dass die Performance im Spiel darunter leidet. Der Grund dafür ist,dass sobald sich ein Polygon in der Darstellungsdistanz befindet, wird es berechnet. Um die Anzahl der Polygone niedrig zu halten, werden die Details als \glqq Objekt-Images \grqq \space gezeichnet. \\\\ 
Bei dem ersten Prototypen entstand das Problem, dass mehrere Spielobjekte zu viele Polygone hatten. Somit mussten die meisten Objekte im Prototypen neu bearbeiten. Dieser Vorgang wird auch als 'Mapping' bezeichnet.
\pagebreak
\subsection{Objekt Mapping}

Bei einem Objekt-Image handelt es sich um ein Bild, welches über das Objekt gelegt wird. Für dieses Verfahren muss erst das Objekt für das Bild \verb+unwrapped+ werden. Dieser Vorgang zerlegt das Objekt in sogenannten Maps.\\\\
In der nächsten Abbildung wird auf der linken Seite eine \verb+unwrapped+ Map von einem Stein gezeigt, der in unseren Prototypen implementiert wird. Auf der rechten Seite ist das 3D-Objekt abgebildet.


\begin{figure}[H]
    \centering
    \includegraphics[width=0.8\textwidth]{chapters/11/Images/StoneAndUnwrap.png}
    \caption{Eine unwrapped map und das dazugehörige Objekt.}
    \label{htl01}
\end{figure}

\noindent Nun könnte man ein Bild unter das entfaltete Mapping legen und damit weiterarbeiten. Auf diesem Bild können nun sämtliche Details eingezeichnet werden. Sobald die Details eingezeichnet wurden, wird das Bild \glqq gebacken \grqq.Das Backen ist der Vorgang wo die berechneten Daten zu einem Bild konventiert werden. Im folgenden Bild sieht man das Objektbild eines Steins: 

\begin{figure}[H]
    \centering
    \includegraphics[width=0.4\textwidth]{chapters/11/Images/SteinColor.png}
    \caption{Eine Colormap von dem Objekt.}
    \label{htl01}
\end{figure}

\subsection{Normal Maps}

Normal-Maps sind Images, die die Eigenschaft besitzen Höhen und Tiefen eines Objektes aufzunehmen. Diese werden auch verwendet, um dem Objekt eine simulierte Höhe und Tiefe zu geben. Bei dem folgendem Bild wird eine Normal-Map von einem Stein gezeigt.

\begin{figure}[H]
    \centering
    \includegraphics[width=0.4\textwidth]{chapters/11/Images/SteinNormal.png}
    \caption{Eine Normalmap von dem Objekt.}
    \label{htl90}
\end{figure}

\noindent Bei der Normal-Map von der Abbildung \ref{htl90} ist deutlich zu erkennen, dass der blaue Bereich die flache Fläche darstellt, während die farblichen Striche hingegen Kratzer zeigen, welche der Stein beinhaltet.

\subsection{Maps Kombinieren}

Als letzter Schritt müssen alle Maps kombiniert werden. Diese werden in dem Grafikprogramm übereinandergelegt.\\\\
Das Endresultat zeigt den Stein mit einer geringe Polygonanzahl aber mit einer großen Vielfalt an Details. In der nächsten Abbildung wird das Endresultat des Steines gezeigt.

\begin{figure}[H]
    \centering
    \includegraphics[width=0.6\textwidth]{chapters/11/Images/SteinCombi.png}
    \caption{Das fertige Objekt.}
    \label{htl01}
\end{figure}

Um das Objekt in Unity zu implementieren, werden alle Maps in Unity gebraucht. Es ist auch wichtig zu bedenken, dass Unity ein anderes Koordinatensystem hat als Blender. Somit muss beim Export die richtige Richtung angegeben werden. Sonst kann es passieren, dass das Objekt in Unity in eine andere Richtung zeigt.

\begin{figure}[H]
    \centering
    \includegraphics[width=0.6\textwidth]{chapters/11/Images/Koordinaten.png}
    \caption{Eine Abbildung unterschiedlicher Programme und deren Koordinatensystemen.}
    \label{htl01}
\end{figure}

\pagebreak
\pagebreak
\setauthorname{Martin Usta}
\chapter{Spieledesign}
\section{Spieledesign}

Das Spieldesign umfasst mehrere Gebiete der Spielentwicklung. Die Gebiete des Spieldesign sind:

\begin{itemize}
    \item Level Design - Umgebungsdesign 
    \item Theme Auswahl 
    \item Spiel Schwierigkeit 
    \item Music 
\end{itemize}

Bei diesen Gebieten der Spielentwicklung ist es wichtig, dass sie einstimmig sind und zueinander passen. Falls die Gebiete des Spiels-design nicht zueinander passen, so wird das Spiele-Erlebnis für den Endkonsumenten schlecht. 

\subsection{Leveldesign - Umgebungsdesign}
Das Leveldesign besteht aus zwei voneinander Abhängigen Gebieten. Das betrifft das Level und die Umgebung die es betrifft. Dabei geht es beim Leveldesign um: Spielfiguren, Plattformen, Levelaufbau, Levelphysik etc. 
Das Umgebungsdesign befasst sich mit der Umgebung des Spiels wie zum Beispiel: Himmel(Skybox), Berge, Flüssigkeiten, Sonneneinstrahlung, etc. 
Beide Aspekte entscheiden wie, realistisch das Spiel ausschaut. Es ist nicht zum Empfehlen den Grafikstyle zwischen den Properties und der Umgebung zu mischen. 

\subsection{Theme Auswahl}
Das Theme bestimmt welche Atmosphäre das Spiel haben soll. Bei dem Theme ist es wichtig das es mit dem Gameplay abgestimmt ist. Es gibt viele Arten von Themes von Horror bis zum Abenteuer ist alles dabei. Nach der Theme Auswahl kann die Planung für die  Game Assets beginnen. Zudem entscheidet das Theme weitere Spieldesign Aspekte: wie die Musik, die Spielmechaniken und die  Story. Wir haben uns für ein Fantasie Theme entschieden. Das heißt das die Grafik sehr bunt gestaltet ist und einen nicht realistischen Grafikstil. Bei einen Fantasie Theme ist die Auswahl der Gestaltung des Levels sehr umfangreich. 

\subsection{Spiel Schwierigkeit}

Bei der Spieleentwicklung ist es wichtig die Schwierigkeit des Spieles zu bestimmen. Schwierigkeit bestimmt den Flow des Spielverlaufs. Die Schwierigkeit sollte je nach Zielgruppe abgestimmt werden. Für die Spieler ist es wichtig, dass das man gefordert wird. Das gibt den User das Gefühl etwas erreicht zu haben. Zudem sollte das Spiel nicht zu schwer sein, damit der Spieler nicht frustriert wird. 


%https://www.nuclino.com/articles/level-design#:~:text=What%20is%20level%20design%3F,player%20and%20keep%20them%20engaged.

\pagebreak
\setauthorname{Martin Usta}
\chapter {Story und Theme}

\subsection{Einführung}
In diesem Kaptitel geht es um die Überlegung und Planung des Themes und der Story. Die Story, oder auch Geschichte, ist die Handlung die während des Spielgeschehens erzählt wird. Das Theme stärkt das Spielgeschehen, indem visuelle und audiodive Spielelemente die Story kräftigt. Bei der Theme Auswahl ist es wichtig, dass diese von der Story bestimmt wird. Das heißt wenn die Story fröhlich ist wird das Theme bunter sein, aber wenn die Story eine traurige Geschichte erzählt, wird das Theme düsterer. 

\subsection{Die Story}
Die Story soll den Spieler während des Aufenthalts in der Spielewelt unterhalten. \hl{Die Story | die story, die story wortwiederholung} kann sowohl fiktiv sein oder realistisch. Ein gutes Beispiel für ein Spiel mit einer realistischen Story ist \verb+Assassin's Creed+. Dieses Spiel erzählt viel über die Vergangenheit. \hl{Da zählt dazu |nicht objektiv, vl unteranderem} wie Davinci seine ersten Erfindungen kreirt bis zu dem Unabhängigkeitstag von Amerika. Doch die meisten Spiele befinden sich in einer fiktiven Welt. Das hat \hl{den Grund, weil} viele Menschen gerne etwas neues sehen wollen. \hl{Es ist wie ein neues Buch lesen mit einer komplett neuen Handlung. |vergleichbar ist es wie das lesen...} Weiteres sollte die Story auch mit dem Entwicklerteam abgesprochen werden. Der Grund dafür ist, dass die Theme gestaltung sonst sehr unpassend \hl{zur ist |?}. 



\subsection{Story telling}

Beim Story telling geht es darum, wie die Geschichte in einem Spiel dem Spieler nahe gebracht wird. Es gibt viele Ansätze, jedoch es gibt zwei Hauptstrategien, welche die heutige Spieleindustrie verwendet. Der eine wäre das environmental Storytelling und der andere das indexcial Storytelling. In dem Artikel \citetitle{Fernadez1} vom \citeauthor{Fernadez1} im Jahr \citeyear{Fernadez1} wird beschrieben die Vor- und Nachteile beider Methoden. Hierbei schrieb der Author \citeauthor{Fernadez1} über das Besprochene in dem Diskurs von der \citefield{Fernadez1}{journaltitle}. Dabei werden einige Aussagen folgender Entwickler zitiert und analysiert (Carson, 2000; Pearce, 2007; Rouse, 2010; Smith and Worch, 2010). In diesem Kapitel werde \hl{ich} genauer erläutern was diese Methoden machen und wann welche eingesetzt wird. 

\subsection{Environmental Storytelling}
Bei dem environmental Storytelling, wird die Geschichte von der Umwelt erzählt. Der Spieler nimmt daher die Rolle eines Besuchers beziehungsweise eines passiven Aktörs in der Handlung. Jedes Ereignis, ist vordefiniert und wird sich nicht ändern. Dabei spielen die Entscheidungen, welcher der Spieler macht keine Rolle. Der Vorteil an dieser Methode ist, dass sich der Spieler komplett auf die Handlung des Spiels konzentrieren kann. Dabei folgt der Spieler einem linearen Weg der Story. Der Nachteil dabei ist, dass sowohl die Story, das Storytelling und das Theme miteinander perfekt synagieren müssen. Zudem gibt environmental Storytelling keine Informationen wie das Spiel funktioniert. Somit muss sich der Spieler selber mit der Steurung befassen. Es sollte nie die Spielewelt mit der echten Welt interagieren.

\subsection{Indexcial Storytelling}
Bei dem indexcial Storytelling wird die Geschichte von Aktionen und Reaktionen des Spielers bestimmt. Hierbei verändert sich die Welt aufgrund des Spielers. In diesem Verfahren ist der Spieler ein aktives Glied in der Story. Ein Beispiel welches \Citeauthor{Fernadez1} erwähnt wäre ein taktischer Shooter wobei der Spieler entscheiden kann, ob er alle Gegner besiegt oder keinen erledigt um an das Ziel zukommen. Hier hat der Spieler eindeutig mehr Freiraumum um das Spielgeschehen.\\\\
Aber auch die Story kann komplett anders erzählt werden als in der environmental Storytelling. Ein gutes Beispiel vom dem Artikel \citetitle{Fernadez1} war das Spiel \verb+Portal+. Hier wurde die Geschichte was vor dem Zerfall des Labors von einen verrückten Wissenschaftler auf die Wände gezeichnet. Sowas nennt man auch Story bites. Dabei wird die Story von Ereignis zu Ereignis erzählt, aber immer nur in kleinen Stücken.
\begin{figure}[H]
    \centering
    \includegraphics[width=0.6\textwidth]{chapters/15/images/Portal.png}
    \caption{Eine Grafik von dem beschrieben Portal Grafitti.}
    \label{UST-6}
\end{figure}

Im Gegensatz zu environmental Stortelling darf das indexcial Storytelling mit dem Spieler \hl{agieren | meinst du interagieren?}. Das bedeutet, dass das Spiel ihm helfen kann weiter zukommen \hl{|mach einen punkt und einen neuen satz aus "indem es den Spieler Tipps gibt." | so wie: Dies passiert über tipps} Die Steuerung kann auch durch ein Tutorial im Spiel erklärt werden. Aber auch durch Schilder oder Bücher die im Spiel verteilt sind können Storyinhalte besitzen. Ein gutes Beispiel wäre \verb+Super Mario 64+ \hl{mach die spiele immer als verb so wie ichs gemacht hab (falls du neue schreibst oder findest)} in dem der Spieler am Anfang ein Schild sieht, worin die Story als auch die Spielsteuerung erklärt wird. Wichtig bei der Methode ist, dass es nicht das Spielerlebnis durch zu viele Interaktionen mit dem Spieler verschlechtert. 

\subsection{Spannung einer Handlung}
Die Handlung eines Spiel sollte immer spannend gehalten werden. Aber der den Spieler nie überfordern. Wenn der Spieler überfordert ist, ist dieser genervt und kann nicht mehr das Spiel genießen. Aber die Story sollte auch nicht zu langweilig sein um das Spielerlebnis zu trüben. Zudem sollte die Handlung eines Spieles der eines Filmes gleichen. Damit ist gemeint, das es am Anfang einen Aufbau der Story gibt. Danach bleibt die Steigung anhaltend bis am Schluss wo jeder Plot aufgelöst wird. Dieses Verfahren wird in fast jedem literarischen Werk verwendet, ob Film, Buch oder sogar Theater Stück. Dieses Schema ist beliebt da es Spannung erzeugt und \hl{den Endverbraucher (Spieler, Leser, Zuschauer) and die Geschichte klammert| den endverbraucher klammert? klingt nicht gut}. 

\subsection{Spannung Gameplay}
Auch das \bettergls{gameplay}{1} muss eine gewisse Spannung haben. Wenn das Spiel zu leicht ist fühlt sich der Spieler unterfordert. Aber wenn das Spiel zu schwer ist er frustriert.\\\\
Wichtig ist das der \bettergls{flow}{2} optimal zum Spiel passt. \hl{Gehen davon} aus es ist ein Spiel für jüngere Verbraucher, dann darf das Spiel nicht zu schwer sein, \hl{da die Erfahrung mit Spielen einfach zu wenig vorhanden ist}. Aber wenn die Zielgruppe Personen sind, die sich mit Videospielen gut auskennen, dann sollten sie auch Herausgefordert werden. Damit sie nicht während des Spielens aufgrund des zu leichten Gameplay keine lust mehr bekommen das Spiel zu spielen. Wichtig ist aber, dass der Flow während des Spielverlaufs ändert. Am Anfang sollte das Spiel einfach gehalten sein. Damit der User einen leichten Start in das Spiel bekommt. Wenn sich der Spieler an die \bettergls{gamemechaniken}{3} gewöhnt hat, muss der Spieler mehr gefordert werden damit sein Interesse bleibt. \hl{Zudem fühlt sich das Spiel eintönig an, wenn immer die gleiche Herausforderung entgegen kommt. | komischer satz} In der folgenden Abbildung aus dem Buch \citetitle{GameDesign} kann man erkennen, dass der Flow sich während des Spiels verändern soll.

\todo{hast du das nicht schonmal geschrieben? bei spiel schwirigkeit?}

\begin{figure}[H]
    \centering
    \includegraphics[width=0.8\textwidth]{chapters/15/images/GameFlow.png}
    \caption{Eine Grafik die den Flow verlauf des Spiels zeigt.}
    \label{UST-4}
\end{figure}


\subsection{Theme}
Das Theme ist dafür zuständig um die visuellen und audiodiven Reize des Spielers zu begeistern. Eine gute Story macht es nicht automatisch zu einem guten Spiel. Das Theme muss zur Story passen. In einem fröhlichen Spiel sind die Farben meist bunt und die Hintergrund-Musik sehr fröhlich und aufmunternd. Nicht wie bei einer düsteren beziehungsweise traurigen Story, da werden eher dunklere Farben und eine langsame umklammernde Musik verwendet. Das Theme muss den Spieler stimmig für die Story machen.

\todo{hast du das mit dem theme nicht auch schonmal geschrieben?}

\subsection{Spielidee für den Prototypen}
Für den Prototypen habe wir (Martin Usta und Lukas Schachinger) uns entschieden das wir einen \bettergls{platformer}{4} machen. \hl{Dieser Sollte beinhalten Sammelobjekte in Form von Münzen.} Aber auch schwierige Passagen in denen der Spieler sich gefordert fühlt. \hl{Der spielbare Charakter sollte auf jeden fall dynamische Animationen besitzen. | well...} Zudem sollten auch Gegner in diesem Spiel vorkommen, um den Spieler zu hindern das Ende zu erreichen. Die Inspiration für diese Art vom Spiel, war das damalige Kult Spiel \verb+Kao the Kangaro Round 2+ welches im November 2004 erschienen ist.\\\\

\subsection{Storyline für den Prototypen}
Unser Hautprotagonist \hl{ generischer Name } ist eingeschlafen und findet sich in eine Traumwelt als Eule wieder. Er muss alles versuchen um aus dieser Welt herauszukommen. Doch diese Aufgabe ist schwieriger als er es vermuten mag. Um dieses Ziel zu bewältigen muss der Spieler Fallen ausweichen, \hl{Rätsel lösen |wirklich?} und Gegener die ihm in den Weg stellen ausschalten. Dafür setzt er seine Eulenfähigkeiten ein. Ob ihm das gelingen wird ist nur eine Frage der Zeit. Aber er weiß, dass er muss so schnell wie möglich aufstehen um sein Kind vom Kindergarten abzuholen.

\begin{figure}[H]
    \centering
    \includegraphics[width=0.8\textwidth]{chapters/15/images/Dreamworld.png}
    \caption{Eine Grafik des ersten Konzepts unsere Traumwelt.}
    \label{UST-7}
\end{figure}



\chapter{User Interface (UI) und Interaktion}

\begin{quote}
\emph{\glqq Im Interface begegnet der Spieler dem Spiel. [...] Die Schnittstelle zwischen Mensch und Computer/Konsole etc.\grqq}~\cite[p.~161]{GameDesign} \\
\end{quote}

Das UI im Spieldesign ist fast der wichtigste Teil der Spielentwicklung, wenn man das Design des eigentlichen Spieles außen vornimmt. 
Ohne eine gute Menu führung und ein gut entworfenes User Interface ist es schwer, dass Spielerinnen und Spieler das Spiel verstehen und spielen können. 

\begin{figure}[H]
    \centering
    \includegraphics[width=0.8\textwidth]{chapters/03/images/Spielinterface.png}
    \caption{Ein Beispiel eines Dialogbaumes von einem Computerspiel.}
    \label{htl01}
\end{figure}

Der abgebildete Dialogbaum zeigt die verschiedenen Menüs und Screens die ein Computerspiel haben kann.
Diese unterscheiden sich in den unterschiedlichen Arten von Spielen. 
Bei einem \gls{multiplayer} Spiel soll die Möglichkeit geboten werden ein Spiel zu erstellen oder einem beizutreten. 
Während es bei einem \gls{singleplayer} wichtig ist Spielstände zu speichern und zu laden.

Das \gls{UI} des Prototyps unterteilt sich in 3 verschiedene Aspekte:

\begin{itemize}
    \item Das Hauptmenü.
    \item Das \gls{UI} während des Spieles.
    \item Das Pause Menu.
\end{itemize}

\noindent
Bei der Entscheidung, welche Komplexität das User Interface des Prototyps haben soll, fiel die Wahl auf ein simples Design. 
Das \gls{UI} soll einerseits die Schlichtheit des eigentlichen Spieles wiederspiegeln und andererseits alle für das Spiel wichtigen Informationen darstellen.

\section{Gestaltung des Hauptmenüs}

Im Folgenden wird das Gestalten der Benutzeroberfläche für das Hauptmenü erläutert. 
Das Hauptmenü in der Spielentwicklung ist vergleichbar mit einem Türvorleger vor einer Haustür. Es soll das Willkommenschild für das eigentliche Spiel sein. 
Mithilfe dieses ersten Eindrucks ist es möglich zu erkennen um welche Art von Spiel es sich handelt. Das \gls{theme} des Spieles spiegelt sich in dem Design des Hintergrundes und in der Schriftart des Hauptmenüs wieder. Die Komplexität steht bei vielen Spielen in direkter Korrelation mit der Anzahl an Einstellungen in dem Hauptmenü.

%Die essenziellen

\begin{figure}[H]
    \centering
    \includegraphics[width=1\textwidth]{chapters/03/images/MainMenu.png}
    \caption{Das Hauptmenü des Prototyps.}
    \label{htl02}
\end{figure}

\section{Game UI und Spielmechanik-Anzeigen}
\section{Optionen und Einstellungen}
\section{Pause-Funktion und Benutzerfreundlichkeit}
\pagebreak
\chapter{Entwicklung des Prototyps}

In diesem Kapitel werden alle Schritte, die in die Kreation des Spiels hineingeflossen sind, beschrieben. Die Entwicklung des Prototypen fing mit der Idee des Themes an und durchlief mehere Phasen bis zur finalen Version des Spiels. Bei jeder dieser Phasen wurden viele Optimierungen gemacht und Fehler ausgebessert. 


\section{Idee und Thema des Spiels}


\begin{description}
  \item [Einführung in die Ausgangsidee des Spiels]
    Wegen des Vorwissens über Unity, welches in der 1 \& 2.ten Klasse unterrichte wurde war die Ausgangsidee für das Spiel war eine einfache Wahl. Die fast einfachste und am weitesten verbreitete Kategorie von Spiel ist ein \gls{jumprun} \footnote[1]{Bedeutung: Spring \& Lauf. Ein Spiel in dem man mit Springen und Laufen ein Ziel erreichen muss.}. 

    \begin{quote}
      \emph{\glqq Ein Merkmal (von Jump'n'Run Spielen) sind die Plattformen. [...] Typisch beim Jump'n'Run ist das Verlieren\grqq}~\cite[1:42-1:57]{ArtOfGaming}
    \end{quote}
    
    Ein Jump'n'Run würde einerseits kreative Freiheit über das Theme geben, aber auch die Möglichkeit bieten viele Aspekte der Spielentwicklung in dieser Diplomarbeit zu präsentieren.

  \item [Beschreibung des gewählten Themas und dessen Bedeutung]
    Da das Spiel nicht an der realen Welt angelegt sein soll, kam die überlegung eine Traumwelt zu bauen. In dieser gäbe es die Freiheit den Charakter, die Gegner und alle anderen Assets in verschiedenen Stilen zu designen. Damit wird auch für die Spieler dieses Prototypen klar das das gewählte Theme Fantasy ist. Diese Themenwahl erleichtert zudem dem Game-Designer die kreative Gestaltung.

  %\item [Erläuterung der Inspirationsquellen und kreativen Einflüsse]
  
\end{description}




\pagebreak



\pagebreak
\chapter{Scripte}

\section{Player Script}

\section{Enemy Script}

\section{Münzen Script}

\pagebreak
\setauthorname{Lukas Schachinger}

\chapter{Fazit}
Das letzte Kapitel dieser Diplomarbeit widmet sich der Frage: \glqq \textbf{Wie entwickelt man ein Computerspiel von Grund auf} \grqq. Während der Arbeit haben wir uns umfassend damit befasst, worauf es bei der Spielentwicklung ankommt. \\

Angefangen hat es mit der Überlegung des grundlegenden Konzeptes des Spiels. Zudem befassten wir uns mit dem Game Designs, der Storytelling-Elemente und der Spielmechaniken. Diese Teile sind das Grundgerüst jedes Spiels und legen den Grundstein für die gesamte Entwicklung.\\

Die technische Umsetzung ist der nächstwichtigste Punkt bei der Kreation eines Spieles. Zuerst muss man sich mit der Auswahl der richtigen Game Engine und Grafikdesigntools auseinandersetzen. Erst dann kann man sich mit der eigentlichen Entwicklung beschäftigen.\\

Wie in dem Kapitel \verb+Entwicklung des Prototyps+ beschrieben kann es passieren, dass mehere Versionen des Spieles erstellt werden muss. Dies ist ein großer Teil bei der Entwicklung eines Spieles. Diese Angehensweise ist hilfreich bei dem Testen oder der Verfeinerung der verschiedenen Aspekte des Spieles. In unserem Beispiel haben wir mehrere Versionen des ersten Levels erstellt um kleine Verbesserungen einzubauen oder bekannte Fehler auszubessern.\\

Zusammenfassend lässt sich sagen, dass die Spielentwicklung sowohl kreativ als auch technisch anspruchsvoll ist. Mit Hingabe und der Bereitschaft zum Lernen können Entwickler wie wir beeindruckende Spielerlebnisse erschaffen. Die sich stetig verändernde Branche eröffnet zudem Möglichkeiten, eigene Visionen zu verwirklichen und sich mit den neuesten Techniken zu beschäftigen.

%\addcontentsline{toc}{chapter}{Spielkonzept} 
%\addcontentsline{toc}{chapter}{Prototyp-Entwicklung}

%\addcontentsline{toc}{chapter}{Tests und Qualitätssicherung?}
%\addcontentsline{toc}{chapter}{Vermarktung und Veröffentlichung}
%\addcontentsline{toc}{chapter}{Zusammenfassung und Ausblick}
% todo: remove contents ^

\newpage
\appendix
\setauthorname{}

\newpage
\printnoidxglossaries

%\input{./chapters/99/99_acknowledgements.tex}

%\addcontentsline{toc}{chapter}{Listings}
%\lstlistoflistings

\addcontentsline{toc}{chapter}{List of Figures}
\listoffigures

\lstlistoflistings

\addcontentsline{toc}{chapter}{Bibliography}
\printbibliography

%\input{./chapters/99_authors.tex}
\chapter{Anhang}
\renewcommand{\arraystretch}{1}
\thispagestyle{empty}
\section{Begleitprotokolle}
\markboth{Begleitprotokolle}{Begleitprotokolle}
\newgeometry{bottom=10mm, left=20mm, right=30mm, top=30mm}

\pagebreak

\subsection{Begleitprotokoll Usta Martin}

\subsubsection{Stundenzahl nach Meilensteinen}

\begin{tabular}{|m{0.3\textwidth}|m{0.6\textwidth}|m{0.1\textwidth}|}
    \hline
    \cellcolor{gray!10} Datum & \cellcolor{gray!10} Meilenstein & \cellcolor{gray!10} Zeit in Stunden \\
    \hline
    9/2022 - 11/2022 & Analyse des Umfelds Projektplanung & 69 \\
    \hline
    11/2022 - 12/2022 & Erstellung des Grundgerüsts für den Prototyp & 69 \\
    \hline
    12/2022 - 02/2023 & Fertigstellung des UI & 69 \\
    \hline
    02/2023 - 05/2023 & Fertigstellung des Prototyps & 69 \\
    \hline
    05/2023 - 09/2023 & Diplomarbeit fertigstellen & 69 \\ 
    \hline
    \multicolumn{2}{|c|}{\cellcolor{gray!30}Gesamtdauer} & 69 \\
    \hline
\end{tabular}

\noindent

\vspace{40pt}

\subsubsection{Stundenerfassung}

\begin{tabular}{|m{0.3\textwidth}|m{0.6\textwidth}|m{0.1\textwidth}|}
    \hline
    \cellcolor{gray!10} Zeitraum & \cellcolor{gray!10} Tätigkeiten & \cellcolor{gray!10} Zeit in Stunden \\
    \hline
    9/2022 - 10/2022 & \hl{Beispiel: } Katalogisieren des Inventars & 69 \\
    \hline
    \multicolumn{2}{|c|}{\cellcolor{gray!30}Gesamtdauer} & 69 \\
    \hline
\end{tabular}

\pagebreak

\subsection{Begleitprotokoll Lukas Schachinger}

\subsubsection{Stundenzahl nach Meilensteinen}

\begin{tabular}{|m{0.3\textwidth}|m{0.6\textwidth}|m{0.1\textwidth}|}
    \hline
    \cellcolor{gray!10} Datum & \cellcolor{gray!10} Meilenstein & \cellcolor{gray!10} Zeit in Stunden \\
    \hline
    9/2022 - 11/2022 & Analyse des Umfelds/Projektplanung & 27 \\
    \hline
    11/2022 - 12/2022 & Erstellung des Grundgerüsts für den Prototyp & 37 \\
    \hline
    12/2022 - 02/2023 & Fertigstellung des UI & 17 \\
    \hline
    02/2023 - 05/2023 & Fertigstellung des Prototyps & 23 \\
    \hline
    05/2023 - 09/2023 & Diplomarbeit fertigstellen & 85 \\ 
    \hline
    \multicolumn{2}{|c|}{\cellcolor{gray!30}Gesamtdauer} & 189 \\
    \hline
\end{tabular}

\noindent

\vspace{40pt}

\subsubsection{Stundenerfassung}

\begin{tabular}{|m{0.206\textwidth}|m{0.7\textwidth}|m{0.094\textwidth}|}
    \hline
    \cellcolor{gray!10} Zeitraum & \cellcolor{gray!10} Tätigkeiten & \cellcolor{gray!10} Zeit in Stunden \\
    \hline
    9/2022 - 9/2022 & Ideensammlung für die Art des Spiels/das Theme/die Story & 8 \\
    \hline
    10/2022 - 11/2022 & Erstellung der Diplomarbeitsvorlage in LaTeX & 12 \\
    \hline
    11/2022 - 11/2022 & Erstellung des Pflichtenheftes & 7 \\
    \hline
    11/2022 - 12/2022 & Erstellung der Dokumentation & 7 \\
    \hline
    12/2022 - 12/2022 & Erstellung des Projektplanes & 5 \\
    \hline
    12/2022- 12/2022 & Recherche der Game Engines & 8 \\
    \hline
    10/2022 - 12/2022 & Entwicklung der ersten Version des Prototypen & 17 \\
    \hline
    12/2022 - 1/2023 & Recherche der wirtschaftlichen Begründung für die Entwicklung eines Spiels & 5 \\
    \hline
    1/2023 - 4/2023 & Entwicklung der zweiten Version des Prototypen & 24 \\
    \hline
    4/2023 - 9/2023 & Entwicklung der finalen Version des Prototypen & 37 \\
    \hline
    8/2023 - 9/2023 & Verfassen der Dokumentation & 32 \\
    \hline
    8/2023 - 9/2023 & Struktur der Dokumentation mit LaTeX anpassen & 8 \\
    \hline
    8/2023 - 9/2023 & Zusammenführung der beiden Dokumentationen des Diplomarbeit Teams & 10 \\
    \hline
    8/2023 - 9/2023 & Diplomarbeit überarbeiten und finalisieren & 9 \\
    \hline
    \multicolumn{2}{|c|}{\cellcolor{gray!30}Gesamtdauer} & 189 \\
    \hline
\end{tabular}


\pagebreak
\section{Betreuungsprotokolle} \markboth{Betreuungsprotokolle}{Betreuungsprotokolle}

% 111111111111111111111111111111111111111111111111111111111111111111111111111

\pagebreak

\noindent
\begin{tabular}{|m{0.2\textwidth}|m{0.6\textwidth}|m{0.2\textwidth}|}
\hline
\raisebox{-0.5\height}{\includegraphics[width=1\linewidth]{media/images/htl-bildung-mit-zukunft.png}} 
&
\begin{center}
{\bfseries\sffamily\small HÖHERE TECHNISCHE BUNDES-LEHR- UND VERSUCHSANSTALT MÖDLING}\\[1ex]
{\small Höhere Lehranstalt für Elektronik und Technische Informatik\\
Kolleg für Informatik}
{\textcolor{gray}{bzw. Aufbaulehrgang für Informatik}}
\end{center} & 
\begin{center}
    {Reife- und Diplomprüfung}
\end{center} \\
\hline
\end{tabular}


\vspace*{20pt}
\subsection*{Betreuungsprotokoll zur Diplomarbeit \hfill lfd. Nr.: 1}
\vspace*{10pt}

\begin{tabular}{m{0.4\textwidth} m{0.4\textwidth}}
\textbf{Themenstellung:} & Unity Game Design und Development \\
\textbf{Kandidaten/Kandidatinnen:} & Schachinger Lukas, Usta Martin \\ \\
\textbf{Jahrgang:} & 2022/23 \\
\textbf{Betreuer/in:} & Hack Niklas \\
\textbf{Ort:} & Mödling \\
\textbf{Datum:} & 30.11.2022\\
\textbf{Zeit:} & 12:30 \\
\end{tabular}

\subsubsection*{Besprechungsinhalt:}
\begin{tabular}{|m{0.2\textwidth}|m{0.8\textwidth}|}
\hline
Name & Notiz \\
\hline
Usta, Schachinger & Anfang der Planung, Planung der Meilensteine, Einarbeitung des nächsten Meilensteins, Einrichten von DevOps Anfang der Umfeldanalyse \\
\hline
Usta & Aufgliederung der Meilensteine in Arbeitspackete,
Zuteilung der Arbeitspackete
Anfang der Umfeldanalyse von Designprogrammen,
2 von 3 wurden analysiert, Auswahlkriterien sind fertig\\
\hline
Schachinger & Aufgliederung der Meilensteine in Arbeitspackete, 
Zuteilung der Arbeitspakete
Anfang der Umfeldanalyse von Game Engines
1 von 3 wurden analysiert, Auswahlkriterien sind fertig\\
\hline
\end{tabular}

\subsubsection*{Aufgaben:}
\begin{tabular}{|m{0.2\textwidth}|m{0.6\textwidth}|m{0.2\textwidth}|}
\hline
Name & Notiz & zu erledigen bis \\
\hline
Usta & Fertigstellen der Umfeldanalysen 
Analyse von Designprogramm ZBrush & 12.12.2022 \\
\hline
Schachinger & Fertigstellen der Umfeldanalysen
Analyse von den Game Engines: Unreal, GoDot & 12.12.2022 \\
\hline
Usta & Grundgerüst Prototyp: 
Graphisches Leveldesign und erstellen der Low-Poly Assets
Implementierung der Charaktersteuerung & 18.12.2022 \\
\hline
Schachinger & Grundgerüst Prototyp: 
Implementierung der Bewegbaren Plattformen & 18.12.2022 \\
\hline
\end{tabular}

% 222222222222222222222222222222222222222222222222222222222222222222222222222

\pagebreak

\noindent
\begin{tabular}{|m{0.2\textwidth}|m{0.6\textwidth}|m{0.2\textwidth}|}
\hline
\raisebox{-0.5\height}{\includegraphics[width=1\linewidth]{media/images/htl-bildung-mit-zukunft.png}} 
&
\begin{center}
{\bfseries\sffamily\small HÖHERE TECHNISCHE BUNDES-LEHR- UND VERSUCHSANSTALT MÖDLING}\\[1ex]
{\small Höhere Lehranstalt für Elektronik und Technische Informatik\\
Kolleg für Informatik}
{\textcolor{gray}{bzw. Aufbaulehrgang für Informatik}}
\end{center} & 
\begin{center}
    {Reife- und Diplomprüfung}
\end{center} \\
\hline
\end{tabular}


\vspace*{20pt}
\subsection*{Betreuungsprotokoll zur Diplomarbeit \hfill lfd. Nr.: 2}
\vspace*{10pt}

\begin{tabular}{m{0.4\textwidth} m{0.4\textwidth}}
\textbf{Themenstellung:} & Unity Game Design und Development \\
\textbf{Kandidaten/Kandidatinnen:} & Schachinger Lukas, Usta Martin \\ \\
\textbf{Jahrgang:} & 2022/23 \\
\textbf{Betreuer/in:} & Hack Niklas \\
\textbf{Ort:} & Mödling \\
\textbf{Datum:} & 19.12.2022 \\
\textbf{Zeit:} & 12:30 \\
\end{tabular}

\subsubsection*{Besprechungsinhalt:}
\begin{tabular}{|m{0.2\textwidth}|m{0.8\textwidth}|}
\hline
Name & Notiz \\
\hline
Usta & ZBrush noch nicht fertig analysiert \\
\hline
Usta & Leveldesign und Low-Poly fertig, Implementierung der Charaktersteuerung noch nicht fertig. \\
\hline
Schachinger & Umfeldanalyse für Unity und Unreal Engine fertig. Umfeldanalyse von GoDot noch nicht fertig. \\
\hline
Schachinger & Implementierung der bewegbaren Plattformen fertig. \\
\hline
\end{tabular}

\subsubsection*{Aufgaben:}
\begin{tabular}{|m{0.2\textwidth}|m{0.6\textwidth}|m{0.2\textwidth}|}
\hline
Name & Notiz & zu erledigen bis \\
\hline
Usta & ZBrush Umfeldanalyse fertigstellen & 28.02.2023 \\
\hline
Usta & Implementierung der Charaktersteuerung fertig & 28.02.2023 \\
\hline
Schachinger & GoDot Umfeldanalyse fertigstellen & 28.02.2023 \\
\hline
Schachinger, Usta & Planung des UI für eine optimale User Experience & 28.02.2023 \\
\hline
Schachinger & Aufbau und Erstellung einer Menu Führung & 28.02.2023 \\
\hline
Schachinger & Erstellung einer Statistikoberfläche & 28.02.2023 \\
\hline
\end{tabular}

% 333333333333333333333333333333333333333333333333333333333333333333333333333

\pagebreak

\noindent
\begin{tabular}{|m{0.2\textwidth}|m{0.6\textwidth}|m{0.2\textwidth}|}
\hline
\raisebox{-0.5\height}{\includegraphics[width=1\linewidth]{media/images/htl-bildung-mit-zukunft.png}} 
&
\begin{center}
{\bfseries\sffamily\small HÖHERE TECHNISCHE BUNDES-LEHR- UND VERSUCHSANSTALT MÖDLING}\\[1ex]
{\small Höhere Lehranstalt für Elektronik und Technische Informatik\\
Kolleg für Informatik}
{\textcolor{gray}{bzw. Aufbaulehrgang für Informatik}}
\end{center} & 
\begin{center}
    {Reife- und Diplomprüfung}
\end{center} \\
\hline
\end{tabular}


\vspace*{20pt}
\subsection*{Betreuungsprotokoll zur Diplomarbeit \hfill lfd. Nr.: 3}
\vspace*{10pt}

\begin{tabular}{m{0.4\textwidth} m{0.4\textwidth}}
\textbf{Themenstellung:} & Unity Game Design und Development \\
\textbf{Kandidaten/Kandidatinnen:} & Schachinger Lukas, Usta Martin \\ \\
\textbf{Jahrgang:} & 2022/23 \\
\textbf{Betreuer/in:} & Hack Niklas \\
\textbf{Ort:} & Mödling \\
\textbf{Datum:} & 28.02.2023 \\
\textbf{Zeit:} & 12:30 \\
\end{tabular}

\subsubsection*{Besprechungsinhalt:}
\begin{tabular}{|m{0.2\textwidth}|m{0.8\textwidth}|}
\hline
Name & Notiz \\
\hline
Usta & ZBrush Umfeldanalyse muss noch überarbeitet werden. \\
\hline
Usta & Implementierung der Charaktersteuerung noch nicht fertiggestellt \\
\hline
Schachinger & GoDot Umfeldanalyse ist fertiggestellt. \\
\hline
Schachinger, Usta & Planung des UI für eine optimale User Experience. UI wurde neu erstellt, muss aber noch überarbeitet werden. \\
\hline
Schachinger & Aufbau und Erstellung einer Menu Führung noch nicht fertig, es fehlen Hauptmenü und Pausemenu.\\
\hline
Schachinger & Erstellung einer Statistikoberfläche fertiggestellt. \\
\hline
\end{tabular}

\subsubsection*{Aufgaben:}
\begin{tabular}{|m{0.2\textwidth}|m{0.6\textwidth}|m{0.2\textwidth}|}
\hline
Name & Notiz & zu erledigen bis \\
\hline
Usta & ZBrush Umfeldanalyse fertigstellen, anpassen auf das Layout & 20.03.2023 \\
\hline
Usta & Implementierung der Charaktersteuerung fertigstellen & 17.04.2023 \\
\hline
Schachinger & Planung des UI für eine optimale User Experience. UI überarbeiten und fertigstellen. & 20.03.2023 \\
\hline
Schachinger & Aufbau und Erstellung einer Menu Führung fertigstellen. & 17.04.2023 \\
\hline
Usta & Implementierung der High-Poly Assets & 31.05.2023 \\
\hline
Usta & Fertigstellung der Designaufgaben & 31.05.2023 \\
\hline
Schachinger & Fertigstellung der Programmieraufgaben & 31.05.2023 \\
\hline
\end{tabular}

% 444444444444444444444444444444444444444444444444444444444444444444444444444

\pagebreak

\noindent
\begin{tabular}{|m{0.2\textwidth}|m{0.6\textwidth}|m{0.2\textwidth}|}
\hline
\raisebox{-0.5\height}{\includegraphics[width=1\linewidth]{media/images/htl-bildung-mit-zukunft.png}} 
&
\begin{center}
{\bfseries\sffamily\small HÖHERE TECHNISCHE BUNDES-LEHR- UND VERSUCHSANSTALT MÖDLING}\\[1ex]
{\small Höhere Lehranstalt für Elektronik und Technische Informatik\\
Kolleg für Informatik}
{\textcolor{gray}{bzw. Aufbaulehrgang für Informatik}}
\end{center} & 
\begin{center}
    {Reife- und Diplomprüfung}
\end{center} \\
\hline
\end{tabular}


\vspace*{20pt}
\subsection*{Betreuungsprotokoll zur Diplomarbeit \hfill lfd. Nr.: 4}
\vspace*{10pt}

\begin{tabular}{m{0.4\textwidth} m{0.4\textwidth}}
\textbf{Themenstellung:} & Unity Game Design und Development \\
\textbf{Kandidaten/Kandidatinnen:} & Schachinger Lukas, Usta Martin \\ \\
\textbf{Jahrgang:} & 2022/23 \\
\textbf{Betreuer/in:} & Hack Niklas \\
\textbf{Ort:} & Mödling \\
\textbf{Datum:} & 20.03.2023 \\
\textbf{Zeit:} & 10:30 \\
\end{tabular}

\subsubsection*{Besprechungsinhalt:}
\begin{tabular}{|m{0.2\textwidth}|m{0.8\textwidth}|}
\hline
Name & Notiz \\
\hline
Usta & ZBrush Umfeldanalyse fertiggestellt und auf ein einheitliches Layout angepasst. \\
\hline
Schachinger & UI noch nicht fertiggestellt Überarbeitung des Overlays (Münzen und Herzen) \\
\hline
\end{tabular}

\subsubsection*{Aufgaben:}
\begin{tabular}{|m{0.2\textwidth}|m{0.6\textwidth}|m{0.2\textwidth}|}
\hline
Name & Notiz & zu erledigen bis \\
\hline
Usta & Implementierung der Charaktersteuerung fertigstellen & 17.04.2023 \\
\hline
Schachinger & Überarbeitung des Overlays (Münzen und Herzen) & 17.04.2023 \\
\hline
Schachinger & Aufbau und Erstellung einer Menu Führung fertigstellen. & 17.04.2023 \\
\hline
Usta & Implementierung der High-Poly Assets & 31.05.2023 \\
\hline
Usta & Fertigstellung der Designaufgaben & 31.05.2023 \\
\hline
Schachinger & Fertigstellung der Designaufgaben & 31.05.2023 \\
\hline
\end{tabular}

% 555555555555555555555555555555555555555555555555555555555555555555555555555

\pagebreak

\noindent
\begin{tabular}{|m{0.2\textwidth}|m{0.6\textwidth}|m{0.2\textwidth}|}
\hline
\raisebox{-0.5\height}{\includegraphics[width=1\linewidth]{media/images/htl-bildung-mit-zukunft.png}} 
&
\begin{center}
{\bfseries\sffamily\small HÖHERE TECHNISCHE BUNDES-LEHR- UND VERSUCHSANSTALT MÖDLING}\\[1ex]
{\small Höhere Lehranstalt für Elektronik und Technische Informatik\\
Kolleg für Informatik}
{\textcolor{gray}{bzw. Aufbaulehrgang für Informatik}}
\end{center} & 
\begin{center}
    {Reife- und Diplomprüfung}
\end{center} \\
\hline
\end{tabular}


\vspace*{20pt}
\subsection*{Betreuungsprotokoll zur Diplomarbeit \hfill lfd. Nr.: 5}
\vspace*{10pt}

\begin{tabular}{m{0.4\textwidth} m{0.4\textwidth}}
\textbf{Themenstellung:} & Unity Game Design und Development \\
\textbf{Kandidaten/Kandidatinnen:} & Schachinger Lukas, Usta Martin \\ \\
\textbf{Jahrgang:} & 2022/23 \\
\textbf{Betreuer/in:} & Hack Niklas \\
\textbf{Ort:} & Mödling \\
\textbf{Datum:} & 18.04.2023 \\
\textbf{Zeit:} & 10:00 \\
\end{tabular}

\subsubsection*{Besprechungsinhalt:}
\begin{tabular}{|m{0.2\textwidth}|m{0.8\textwidth}|}
\hline
Name & Notiz \\
\hline
Usta & Implementierung der Charaktersteuerung noch nicht fertiggestellt. \\
\hline
Schachinger & Überarbeitung des Overlays (Münzen und Herzen) fertiggestellt. \\
\hline
Schachinger & Pause und Quit Menu fertig, Spiel Hauptmenu noch nicht erstellt. \\
\hline
\end{tabular}

\subsubsection*{Aufgaben:}
\begin{tabular}{|m{0.2\textwidth}|m{0.6\textwidth}|m{0.2\textwidth}|}
\hline
Name & Notiz & zu erledigen bis \\
\hline
Schachinger & Spiel Hauptmenu erstellen. & 31.05.2023 \\
\hline
Usta & Charaktersteuerung muss überarbeitet werden. (Fähigkeiten und Text weiter verfassen) & 31.05.2023 \\
\hline
Usta & Implementierung der High-Poly Assets & 31.05.2023 \\
\hline
Usta & Fertigstellung der Designaufgaben & 31.05.2023 \\
\hline
Schachinger & Fertigstellung der Programmieraufgaben & 31.05.2023 \\
\hline
\end{tabular}

% 666666666666666666666666666666666666666666666666666666666666666666666666666

\pagebreak

\noindent
\begin{tabular}{|m{0.2\textwidth}|m{0.6\textwidth}|m{0.2\textwidth}|}
\hline
\raisebox{-0.5\height}{\includegraphics[width=1\linewidth]{media/images/htl-bildung-mit-zukunft.png}} 
&
\begin{center}
{\bfseries\sffamily\small HÖHERE TECHNISCHE BUNDES-LEHR- UND VERSUCHSANSTALT MÖDLING}\\[1ex]
{\small Höhere Lehranstalt für Elektronik und Technische Informatik\\
Kolleg für Informatik}
{\textcolor{gray}{bzw. Aufbaulehrgang für Informatik}}
\end{center} & 
\begin{center}
    {Reife- und Diplomprüfung}
\end{center} \\
\hline
\end{tabular}


\vspace*{20pt}
\subsection*{Betreuungsprotokoll zur Diplomarbeit \hfill lfd. Nr.: 6}
\vspace*{10pt}

\begin{tabular}{m{0.4\textwidth} m{0.4\textwidth}}
\textbf{Themenstellung:} & Unity Game Design und Development \\
\textbf{Kandidaten/Kandidatinnen:} & Schachinger Lukas, Usta Martin \\ \\
\textbf{Jahrgang:} & 2022/23 \\
\textbf{Betreuer/in:} & Hack Niklas \\
\textbf{Ort:} & Mödling \\
\textbf{Datum:} & 31.05.2023 \\
\textbf{Zeit:} & 12:00 \\
\end{tabular}

\subsubsection*{Besprechungsinhalt:}
\begin{tabular}{|m{0.2\textwidth}|m{0.8\textwidth}|}
\hline
Name & Notiz \\
\hline
Usta & Charaktersteuerung muss noch überarbeitet werden. \\
\hline
Schachinger & Hauptmenu erstellt und implementiert. Settings müssen noch überarbeitet werden. \\
\hline
Usta & High Poly Assets erstellt muss noch implementiert werden. \\
\hline
Usta & Dokumentation über die Spiel Optimierung muss ausgebessert werden. \\
\hline
Schachinger & Programmieraufgaben sind noch nicht fertiggestellt. \\
\hline
Usta & Designaufgaben sind noch nicht fertiggestellt. \\
\hline
\end{tabular}

\subsubsection*{Aufgaben:}
\begin{tabular}{|m{0.2\textwidth}|m{0.6\textwidth}|m{0.2\textwidth}|}
\hline
Name & Notiz & zu erledigen bis \\
\hline
Schachinger & Settings im Pause Menu und im Haupt Menu überarbeiten. & 20.06.2023 \\
\hline
Usta & Charaktersteuerung muss weiter überarbeiten.  & 20.06.2023 \\
\hline
Usta & Implementierung der High-Poly Assets & 20.06.2023 \\
\hline
Usta & Fertigstellung der Designaufgaben & 20.06.2023 \\
\hline
Schachinger & Fertigstellung der Programmieraufgaben & 20.06.2023 \\
\hline
\end{tabular}

% 777777777777777777777777777777777777777777777777777777777777777777777777777

\pagebreak

\noindent
\begin{tabular}{|m{0.2\textwidth}|m{0.6\textwidth}|m{0.2\textwidth}|}
\hline
\raisebox{-0.5\height}{\includegraphics[width=1\linewidth]{media/images/htl-bildung-mit-zukunft.png}} 
&
\begin{center}
{\bfseries\sffamily\small HÖHERE TECHNISCHE BUNDES-LEHR- UND VERSUCHSANSTALT MÖDLING}\\[1ex]
{\small Höhere Lehranstalt für Elektronik und Technische Informatik\\
Kolleg für Informatik}
{\textcolor{gray}{bzw. Aufbaulehrgang für Informatik}}
\end{center} & 
\begin{center}
    {Reife- und Diplomprüfung}
\end{center} \\
\hline
\end{tabular}


\vspace*{20pt}
\subsection*{Betreuungsprotokoll zur Diplomarbeit \hfill lfd. Nr.: 7}
\vspace*{10pt}

\begin{tabular}{m{0.4\textwidth} m{0.4\textwidth}}
\textbf{Themenstellung:} & Unity Game Design und Development \\
\textbf{Kandidaten/Kandidatinnen:} & Schachinger Lukas, Usta Martin \\ \\
\textbf{Jahrgang:} & 2022/23 \\
\textbf{Betreuer/in:} & Hack Niklas \\
\textbf{Ort:} & Mödling \\
\textbf{Datum:} & 20.06.2023 \\
\textbf{Zeit:} & 13:00 \\
\end{tabular}

\subsubsection*{Besprechungsinhalt:}
\begin{tabular}{|m{0.2\textwidth}|m{0.8\textwidth}|}
\hline
Name & Notiz \\
\hline
Schachinger & Menüführung muss noch finalisiert werden. \\
\hline
Schachinger & Finalisierung der Programmieraufgaben. \\
\hline
Usta & Charaktersteuerung muss noch finalisiert werden. \\
\hline
Usta & High-Poly Assets müssen noch fertiggestellt werden. \\
\hline
Usta & Finalisierung der Designaufgaben. \\
\hline
\end{tabular}

\subsubsection*{Aufgaben:}
\begin{tabular}{|m{0.2\textwidth}|m{0.6\textwidth}|m{0.2\textwidth}|}
\hline
Name & Notiz & zu erledigen bis \\
\hline
Schachinger & Fertigstellung der Menüführung. & 01.09.2023 \\
\hline
Schachinger & Finalisierung der Programmieraufgaben. & 01.09.2023 \\
\hline
Usta & Finalisierung der Charaktersteuerung.  & 01.09.2023 \\
\hline
Usta & Fertigstellung der High-Poly Assets. & 01.09.2023 \\
\hline
Usta & Finalisierung der Designaufgaben. & 01.09.2023 \\
\hline
\end{tabular}

\end{document}